\documentclass[preprint, 12pt]{elsarticle}
\usepackage{amssymb}
\usepackage{amsfonts}
\usepackage{amsmath}
\usepackage{color}
\usepackage{ulem}
\usepackage[portuguese, english]{babel}
\usepackage[utf8]{inputenc}
\usepackage[T1]{fontenc}

\newcommand{\overbar}[1]{\mkern 1.5mu\overline{\mkern-1.5mu#1\mkern-1.5mu}\mkern 1.5mu}


\journal{Physics Letters B}

\begin{document}

\begin{frontmatter}

\title{$\gamma^* \gamma$, $\gamma^* p$ and $pp$ scattering in Improved Holographic QCD}

\author[ath]{Artur Amorim~\corref{cor1}}
\author[ath]{Miguel S. Costa~\corref{cor1}}
\address[ath]{Centro de F\'{\i}sica do Porto e Departamento de F\'{\i}sica e Astronomia da Faculdade de Ci\^encias da Universidade do Porto, Rua do Campo Alegre 687, 4169-007 Porto, Portugal}

\cortext[cor1]{Corresponding author}

\begin{abstract}

This document is a set of notes that I have elaborated while using the holographic model developed by us to fit $\sigma\left(\gamma \gamma \rightarrow X\right)$, $F_2^\gamma\left(x, Q^2\right)$ and $\sigma\left(\gamma p \rightarrow X\right)$ data. The kernel parameters of the holographic models are fixed by the structure functions $F_2$ and $F_L$ of the proton while the fitting parameters are related to the gravitational couplings that appear in the vertices of the Witten diagram.

\end{abstract}

\end{frontmatter}

\section{$\gamma^{*}\gamma$ observables}
\begin{figure}[!h]
  \center
  \includegraphics[height=4cm]{images/Witten_diagram} 
  \caption{Tree level Witten diagram representing spin $J$    exchange in a $12\to34$ scattering. 
%     Double lines are used to denote graviton's propagator. The $n_1$ and $n_3$ labels denote incoming/outgoing proton's polarizations respectively, for the forward scattering $n_1=n_3$.
}
  \label{fig:Witten_diagram}
\end{figure}

In high-energy $e^{+}e^{-}$ interactions some electrons and positrons are scattered by emitting virtual photons whose virtualities can be computed by measuring the angles and energies of the scattered electrons and positrons. The virtual photons can then fluctuate into quark-anti-quark paris and hence generate a hadronic final state $X$. It is in this context that we refer to $\gamma^{*} \gamma^{*}$ scattering.

%Then In the classical theory of electromagnetism two photons can not scatter since they don't have electric charge. However, in the context of Quantum Electrodynamics (QED) it is known that the photons can fluctuate into a fermion-anti-fermion pair and scattering can proceed. The fermion pair can either be leptons or quarks. If the pair is made of leptons the process can be calculated in QED. If a pair of quarks is produced the process is more complex due to the bigger spectrum of fluctuations as well QCD corrections. 

The hadronic final state is produced by the virtual photons fluctuating into the quark-anti-quark pairs. Here the photon reveals its point-like and/or its hadron-like behaviour. In the point-like case one of the quarks takes part in the hard interaction while in the hadron-like case the photons fluctuate into hadrons with the same quantum numbers of the photon (i.e. vector mesons like $\rho$, $\omega$, $\phi$) and the interaction is the same as in hadron-hadron scattering. These pictures coexist together and are dominant in different kinematical regions. For high transverse momentum of the quarks or high virtuality of one of the photons the point-like nature is dominant. For lower values of the the photon virtuality the interaction is spread over longer times giving time for the quarks to form bound states through gluon exchange.

The process $e^{+}e^{-} \rightarrow e^{+}e^{-} X$ is factorised into two three terms: one for the radiation of  the virtual photon from the electron, one for the radiation of the other virtual photon from the positron and the term that couples the $\gamma^{*} \gamma^{*}$ system to the final hadronic state $X$. Like we do in electron-proton \textit{Deep Inelastic Scattering} we can define the following variables
\begin{align}
	&y_1 = \frac{q_1 \cdot k_2}{k_1 \cdot k_2} \qquad y_2 = \frac{q_2 \cdot k_1}{k_1 \cdot k_2} \\
	& x_1 = \frac{Q_1^2}{2 q_1 \cdot k_2} \qquad x_2 = \frac{Q^2_2}{2 q_2 \cdot k_1},
\end{align}
where $k_1$ and $k_2$ are the four-momentum of the incoming leptons, $q_1$ and $q_2$ the four-momentum of the two off-shell photons and $Q_i^2 = -q_i^2, \quad i = 1, 2$ are the photon virtualities. In terms of these variables we can write the differential cross-section for $e^{+}e^{-} \rightarrow e^{+}e^{-} X$ as~\cite{donnachie_dosch_landshoff_nachtmann_2002}
\begin{align}
\frac{d^4 \sigma}{d y_1 d y_2 d Q_1^2 d Q_2^2} &= {\left( \frac{\alpha}{2 \pi} \right)}^2 \left[ P^{T}_{\gamma/e^{-}} \left(y_1, Q_1^2\right) P^{T}_{\gamma/e^{+}} \left(y_2, Q_2^2\right) \sigma^{TT} \left(Q_1^2, Q_2^2, W^2\right) \right. \notag \\
&\left. + P^{T}_{\gamma/e^{-}} \left(y_1, Q_1^2\right) P^{L}_{\gamma/e^{+}} \left(y_2\right) \sigma^{TL} \left(Q_1^2, Q_2^2, W^2\right) \right. \notag \\
&\left. + P^{L}_{\gamma/e^{-}} \left(y_1\right) P^{T}_{\gamma/e^{+}} \left(y_2, Q_2^2\right) \sigma^{LT} \left(Q_1^2, Q_2^2, W^2\right) \right. \notag \\
&\left.  + P^{L}_{\gamma/e^{-}} \left(y_1\right) P^{L}_{\gamma/e^{+}} \left(y_2\right) \sigma^{LL} \left(Q_1^2, Q_2^2, W^2\right) \right] \frac{1}{Q_1^2 Q_2^2}
\end{align}
where the functions $P^T_{\gamma / e}$ and $P^L_{\gamma / e}$ are related to flux factors and are given by
\begin{align}
&P^{T}_{\gamma/e} \left(y, Q^2\right) = \frac{1+{\left(1-y\right)}^2}{y} - \frac{2 m_e^2 y}{Q^2} \notag \\
&P^{L}_{\gamma/e} \left(y\right) = 2\frac{1-y}{y}.
\label{eq:ee_eeX_sigma}
\end{align}
The cross sections $\sigma^{ij}\left(Q_1^2, Q_2^2, W^2\right)$ are the total cross sections for $\gamma^*\left(Q_1^2\right) \gamma^{*}\left(Q_2^2\right) \rightarrow {\rm X}$ for incoming photons with transverse (T) or longitudinal (L) in their centre-of-mass frame. 
The differential cross-section expression also results from integrating over the angle between the plane of the scattered leptons in the center-of-mass frame of the $\gamma^* \gamma^*$ system and does not include terms that are present for polarized lepton beams.

The observables we want to study can be expressed in terms of the cross-sections $\sigma^{ij}\left(Q_1^2, Q_2^2, W^2\right)$. Hence we will compute these cross-sections holographically by starting to compute the general scattering amplitude of $\gamma^{*}\left(Q_1\right)\gamma^{*}\left(Q_2\right) \rightarrow \gamma^{*}\left(Q_3\right) \gamma^{*}\left(Q_4\right)$ for arbitrary polarisation of the incoming and outgoing virtual photons. Later, from this general amplitude, we specify for the case of $\sigma\left(\gamma \gamma \rightarrow X\right)$ and $F_2^\gamma\left(x, Q^2\right)$. The Witten diagram that is relevant for our calculations is the one in figure~\ref{fig:Witten_diagram} (this needs to be changed of course).

\subsection{Kinematics}
We start with kinematics first. Consider light-cone coordinates~$\left(+, -, \perp \right)$ with metric given by $ds^2 = - dx^{+} dx^{-} + d x_\perp^2$, where $x_\perp$ is a vector of the impact parameter space $\mathbb{R}^2$. The incoming photons have space-like momenta
\begin{equation}
k_1 = \left( \sqrt{s}, - \frac{Q_1^2}{\sqrt{s}}, 0 \right), \quad k_2 = \left( - \frac{Q_2^2}{\sqrt{s}}, \sqrt{s},  0 \right),
\end{equation}
while the outgoing off-shell photons have also the space-like momenta
\begin{equation}
k_3 = - \left( \sqrt{s},  \frac{q_\perp^2 - Q_3^2}{\sqrt{s}}, q_\perp \right), \quad k_4 = - \left(  \frac{q_\perp^2-Q_4^2}{\sqrt{s}}, \sqrt{s},  -q_\perp \right).
\end{equation}
The off-shellness of these 4-vectors is $k_i^2 = Q_i^2 > 0 \, \left( i = 1, \dots, 4\right)$. We consider the Regge limit of large s and fixed $t = -q_\perp^2$.

We can now define the photons polarization vectors $n^\lambda_i$ that satisfy the condition
\begin{equation}
n_i^\lambda \cdot k_i = 0.
\end{equation}
The incoming photons have the following polarizations
\begin{equation}
    n_1=
    \begin{cases}
      \left(0,0,1,0\right), & \lambda=1 \\
      \left(0,0,0,1\right), & \lambda=2 \\
      \frac{1}{Q_1} \left( \sqrt{s}, \frac{Q_1^2}{\sqrt{s}}, 0, 0 \right), & \lambda = 3
    \end{cases}
\end{equation}
\begin{equation}
    n_2=
    \begin{cases}
      \left(0,0,1,0\right), & \lambda=1 \\
      \left(0,0,0,1\right), & \lambda=2 \\
      \frac{1}{Q_2} \left( \frac{Q_2^2}{\sqrt{s}}, \sqrt{s}, 0, 0 \right), & \lambda = 3
    \end{cases}
\end{equation}  
and the outgoing off-shell photons have the following pollarization vectors
\begin{equation}
    n_3=
    \begin{cases}
      \left(0,\frac{2 q_x}{\sqrt{s}},1,0\right), & \lambda=1 \\
      \left(0,\frac{2 q_y}{\sqrt{s}},0,1\right), & \lambda=2 \\
      \frac{1}{Q_3} \left(\sqrt{s}, \frac{Q_3^2+q_\perp^2}{\sqrt{s}}, q_\perp \right), & \lambda = 3
    \end{cases}
\end{equation}  
\begin{equation}
    n_4=
    \begin{cases}
      \left(-\frac{2 q_x}{\sqrt{s}},0,1,0\right), & \lambda=1 \\
      \left(-\frac{2 q_y}{\sqrt{s}},0,0,1\right), & \lambda=2 \\
      \frac{1}{Q_4} \left(\frac{Q_4^2+q_\perp^2}{\sqrt{s}},\sqrt{s},- q_\perp \right), & \lambda = 3
    \end{cases}
\end{equation}
Notice that the transverse photons~$\left(\lambda=1,2\right)$ are normalized such that~$n^2 = 1$ while the longitudinal photons~$\left(\lambda=3\right)$ are normalized such that~$n^2 = -1$

\subsection{External states and Spin J fields}

The off-shell photons the external states are sources of the conserved U(1) current $j^\mu = \psi \gamma^\mu \psi$, where quark field $\psi$ is associated with the open string sector. According to the gauge/gravity duality dictionary, these current operators are dual to the non-normalizable modes of a massless~$U(1)$ gauge field A. We assume that this field comes from the open string sector and that it has a minimal coupling to gravity. Then its equations of motion can be derived from the action 
\begin{equation}
S_A = - \frac{1}{4} \int d^5 X \sqrt{-g} e^{- \Phi} F_{ab} F^{ab}
\end{equation}
where $F=dA$ and we use the notation $X^a=(z,x^\alpha)$ for five-dimensional points. 
We will fix the gauge of the $U(1)$ bulk field to be $D_a A^a = 0$, which gives  
$A_z = 0$ and $\partial_\mu A^\mu = 0$. In this gauge, the solution to the equation of motion 
$ \nabla_a \left( e^{-\Phi} F^{ab} \right)=0$ is given by
\begin{equation}
 A_\mu^\lambda \left( X ; k\right) =  n_\mu^\lambda \, f_k ( z )\,e^{i k \cdot x}\,,
\end{equation}
where $f_k(z)$ solves the differential equation
\begin{equation}
  \label{eq:u1 eom gauge fixed}
  \left[-Q^2+e^{\Phi-A}\partial_z\left(e^{A-\Phi}\partial_z \right) \right]f_Q(z)= 0 \,.
\end{equation}
The momentum  $k$ and the polarisation vector $n^\lambda$ satisfy
\begin{equation}
  k^2 = Q^2 \,, \qquad 
  n^\lambda_z = 0 \,, \qquad
   k \cdot n^\lambda = 0 \,,
\end{equation}
where the boundary polarisations are given by the expressions of the last section. 
We choose as UV boundary condition $f(0)=1$ which gives the non-normalizable solution.
Later it will be useful to use the identities
\begin{align}
&F_{\mu\nu}\left(X; k, n\right) = 2 i k_{[\mu}n_{\nu]} f_Q \left(z\right) e^{i k \cdot x}, \notag \\
&F_{z\nu} \left(X; k, n\right) = n_\nu \dot{f}_Q e^{i k \cdot x}.
\label{eq:F_expression}
\end{align}

We will consider the holographic $F_2$ and $F_L$ structure functions of the proton and the cross-sections $\sigma\left(\gamma p \rightarrow X\right)$ and $\sigma\left(p p \rightarrow X\right)$ and hence we need to have a holographic external state for the proton.  The unpolarized proton will be dual to a scalar field $\Upsilon$ whose normalizable mode is of the form 
\begin{equation}
\Upsilon(X;p )= \upsilon_m(z) \, e^{i p\cdot x} \,,
\label{eq:proton}
\end{equation}
where $p$ is the momentum and $m^2=-p^2$.
As explained in detail in~\cite{ballon_bayona_unity_2017},
the specific details of the function $\upsilon_m(z)$ will not be important for it appears inside an integral that can be absorbed in a fitting parameter to be defined later.

The spin $J$ fields in the Witten diagram are dual to the twist two operators and their exchange dominates at low $x$ in DIS processes. In~\cite{ballon_bayona_unity_2017}  it was shown that the equation
\begin{align}
&\left( \nabla^2 - 2 e^{-2A} \dot{\Phi} \nabla_z - \frac{\Delta\left(\Delta - 4\right)}{L^2} + J \dot{A}^2 e^{-2A} + \right. \notag \\
&\left. + \left(J-2\right) e^{-2A} \left( a \ddot{\Phi} + b \left(\ddot{A}-\dot{A}^2\right) + c \dot{\Phi}^2 \right)\right) h^{TT}_{\alpha_1 \cdots \alpha_J} = 0 \, ,
\end{align}
with
\begin{equation}
\frac{\Delta\left(\Delta - 4\right)}{L^2}  = \frac{2}{l_s^2} \left( J - 2 \right) \left( 1+ \frac{d}{\lambda} \right) + \frac{1}{\lambda^{4/3}} \left(J^2 - 4 \right)
\end{equation}
was a good phenomenological equation for the TT components of the spin $J$ fields when considering data of the proton structure function $F_2\left(x, Q^2\right)$. 
\begin{table}[t!]
 \centering
  \caption{Values of the parameters that appear in the equation of motion of the spin $J$ field. They were fixed by fitting the data of the proton structure functions $F_2\left(x,Q^2\right)$ and $F_L\left(x, Q^2\right)$ in~\cite{gluonPDF_IHQCD_2020}. All parameters are dimensionless except for $[l_s]=L$, whose inverse numerical value is expressed in ${\rm GeV}$ units. The intercept of the first four Pomeron trajectories are also shown}
\label{table:pomeron_kernel_parameters}
\vspace{0.5cm}
  \begin{tabular}{|c|c|c|}
  \hline
  parameter & value  & Intercept  \\ \hline
  $l_s^{-1}$  & 6.491 & $j_0 = 1.17$   \\ \hline
  a           & -4.567  & $j_1 =1.08$   \\ \hline
  b           & 1.485  & $j_2 =0.975$     \\ \hline
  c           & 0.653  & $j_3 =0.913$    \\ \hline
  d           & -0.113 &  \\ \hline
  \end{tabular}
  \end{table}
In this work we will consider the same equation of motion but with the parameters given in table~\ref{table:pomeron_kernel_parameters}. These parameters were obtained in~\cite{gluonPDF_IHQCD_2020} by fitting the experimental data of the proton structure functions $F_2\left(x,Q^2\right)$ and $F_L\left(x, Q^2\right)$. In the Regge limit, due to the kinematics of the external states, we will only be interested in the component $\Pi_{+\cdots+,-\cdots -}\left(X, \bar{X}\right)$ of the spin $J$ propagators. This component satisfies the identity
\begin{align}
\label{eq:propagator_identity}
&\int d^2l_\perp e^{-i q_\perp \cdot l_\perp} \int \frac{dw^+ dw^-}{2} \Pi_{+\dots+, - \dots -}\left(w^+, w^-, l_\perp, z, \bar{z}\right) = \notag \\
&- \frac{i}{2^J} {\left( e ^{A+\bar{A}}\right)}^{J-1} G_J \left(z, \bar{z}, t\right) \, , \\
& G_J\left(z, \bar{z}, t\right) = e^{\Phi + \bar{\Phi} - \frac{A + \bar{A}}{2}} \sum_n \frac{\psi_n \left(J, z\right) {\psi_n \left(J, \bar{z}\right)}^*}{t_n \left(J\right) - t}
\end{align}
where $w = \left(w^+, w^-, l_\perp\right)$ is the difference between the Minkowski indices of the bulk points $X$ and $\bar{X}$.
The functions $\psi_n\left(J, z\right)$ are the normalisable modes of the spin $J$ field and are eigenfunctions of the Schr\"{o}dinger equation
\begin{equation}
  \left(-\frac{d^2}{dz^2}+U_J(z)\right)\psi_n(J,z)=t_n\left(J\right)\psi_n(J,z)\,,
\end{equation}
with eigenvalues $t_n \left(J\right)$. The associated Schr\"{o}dinger potential is defined by
\begin{align}
 & U_J(z)= \frac{3}{2}\left(\ddot A - \frac{2}{3}\ddot \Phi\right) + \frac{9}{4}{\left(\dot A - \frac{2}{3}\dot \Phi \right)}^2 + \\
 &+ (J - 2)e^{-2A}\bigg[\frac{2}{l^2_s}\left(1+\frac{d}{\sqrt{\lambda}}\right) + \frac{J + 2}{\lambda^{4/3}}+e^{2A} \left(a \ddot \Phi + b \left(\ddot A - \dot{A}^2\right) + c \dot{\Phi}^2  \right)\bigg]\,.
 \nonumber
\end{align}
where the first line represents the potential for the graviton while the other terms are deformations of it. Later, in order to perform the Sommerfeld-Watson transform in Regge theory, we will consider the analytical continuation of this potential and the value of the intercept $J = j_n$ is obtained as the  $n$-th  eigenvalue that solves the equation $t_n(J)=0$. This eigenvalue problem is solved with a Chebyschev pseudospectral method with 1000 points. The value of the intercepts for the values of $l_s$, $a$, $b$, $c$ and $d$ that we use are also presented in table~\ref{table:pomeron_kernel_parameters}. 
The parameters of the spin $J$ field equation can be interpreted in terms of low energy effective string theory. $l_s$ is the string length; $a$, $b$ and $c$ come from a first order derivative expansion in effective field theory; $d$ is related to the anomalous dimension $\gamma_J$ of the twist 2 operators, or it can also be thought as a description of how the masses of the closed strings excitations are corrected in backgrounds with small curvature.

Finally, in order to compute the Witten diagram we also need to know the external scattering states couple with the spin J fields $h_{a_1 \dots a_J}$ in the graviton Regge trajectory. 
Furthermore, in the Regge limit we will only be interested in the TT components $h_{\alpha_1 \dots \alpha_J}$ of these fields. The derivation of this coupling for the $U\left(1\right)$ gauge and scalar fields has been done in~\cite{ballon_bayona_unity_2017} and hence we only present the results
\begin{align}
 &\kappa_J  \int  d^5 X \sqrt{-g} \, e^{-\Phi }   F^{\alpha_1 a} \partial^{\alpha_2} \dots \partial^{\alpha_{J-1}}F^{\alpha_J}_{\ \ a} h_{\alpha_1\dots \alpha_J}, \\
 \label{eq:scalar_spin_J_coupling}
 &\bar{\kappa}_J  \int  d^5 X \sqrt{-g} \, e^{-\Phi }   \Upsilon \partial^{\alpha_1} \dots \partial^{\alpha_{J}} \Upsilon h_{\alpha_1\dots \alpha_J} .
\end{align}

\subsection{Computation of the Witten diagram}
The amplitude associated with the exchange of a spin J field in the Witten diagram is given by
\begin{align}
&A_J \left(k_i, \lambda_j\right) = - k_J^2 \int d^5X d^5 \bar{X} \sqrt{-g} \sqrt{-\bar{g}} e^{-\Phi} e^{-\bar{\Phi}} \times  \notag \\ 
&F_{- a}^{(1)} \left(k_1, z\right) \partial_{-}^{J-2}F^{a \, (3)}_{-} \left(k_3, z \right)  F_{+ b}^{(2)} \left(k_2, \bar{z}\right) \bar{\partial}_{+}^{J-2}F^{b \, (4)}_{+} \left(k_4, \bar{z} \right) \times \notag \\
& \Pi^{- \dots - , + \dots +}\left(X, \bar{X}\right)
\end{align}
Using the kinematics introduced and lowering the indeces of the bulk-to-bulk propagator of the spin J field we get
\begin{align}
&- k_J^2 {\left( \frac{i}{2} \sqrt{s}\right)}^{2J-4} \int d^5X d^5\bar{X} \frac{e^{5 (A+\bar{A}) }}{4} e^{-\Phi - \bar{\Phi}} {\left(4 e^{-2(A+\bar{A})}\right)}^J e^{-i q_\perp \cdot \left(x_\perp - \bar{x}_\perp\right)} \times \notag \\
& \times F_{-a}^{(1)}F^{a \, (3)}_{-} \Pi_{+\dots+, - \dots -}\left(X, \bar{X}\right)  F_{+ b}^{(2)}F^{b \, (4)}_{+}
\end{align}
By performing the change of variables $w = x - \bar{x}$, using the propagator identity in equation (\ref{eq:propagator_identity}) we get
\begin{align}
&\frac{i V k_J^2}{2^J} s^J \int dz d\bar{z} e^{3\left(A+\bar{A}\right)} e^{- \Phi - \bar{\Phi}} e^{-(1+J)(A+\bar{A})} F(1,3)F(2,4) G_J \left(z, \bar{z}, t\right), \notag \\
& F(i,j) = 
\begin{cases}
f_{Q_i} f_{Q_j} , \lambda_i = \lambda_j = 1,2 \\
\frac{\dot{f_{Q_i}}}{Q_i} \frac{\dot{f_{Q_j}}}{Q_j}, \lambda_i = \lambda_j = 3
\end{cases}, \notag \\
&G_J \left(z, \bar{z}, t\right) = e^{B+\bar{B}} \sum_n \frac{\psi_n\left(J, z\right) {\psi_n \left(J, \bar{z}\right)}^*}{t_n\left(J\right) - t},
\end{align}
where $B = \Phi - A/2$ in IHQCD.

We now need to sum the above amplitude over the even spin-J fields with $J>2$. Such sum can be computed through a Sommerfeld-Watson transform
\begin{equation}
\frac{1}{2} \sum_J s^J + (-s)^J = -\frac{\pi}{2} \int \frac{dJ}{2 \pi i} \frac{s^J + (-s)^J}{\sin \pi J}.
\end{equation}
This assumes that an analytic continuation of the amplitude to the complex J-plane is possible. We now deform the integral from the poles at even J, to the poles $J = j_n\left(t\right)$ defined by $t_n(J) = t$. The scattering domain of negative t contains these poles along the real axis for $J<2$. The forward scattering amplitude ($t= 0$) of $\gamma^{*}\left(Q_1\right)\gamma^{*}\left(Q_2\right) \rightarrow \gamma^{*}\left(Q_3\right) \gamma^{*}\left(Q_4\right)$ is then
\begin{align}
&\mathcal{A}^{\lambda_1, \lambda_2, \lambda_3, \lambda_4}_{Q_1, Q_2, Q_3, Q_4}\left(s,t = 0\right) = - \frac{\pi}{2} \sum_n s^{j_n\left(t\right)} \left( i + \cot\left(\frac{\pi j_n\left(0\right)}{2}\right) \right) \frac{k_{j_n\left(0\right)}^2}{2^{j_n\left(0\right)}} \frac{d j_n}{dt} \times \notag \\
& \times \int dz e^{-\left(j_n - 3/2\right) A} F(1,3) \psi_n\left(z\right) \int d\bar{z} e^{-(j_n-3/2)\bar{A}} F(2,4) {\psi_n\left(\bar{z}\right)}^*
\label{eq:scattering_amplitude}
\end{align}

The total cross-sections $\sigma^{ij}\left(Q_1^2, Q_2^2, W^2\right)$ can now be computed using the optical theorem and appropriate photon polarisations as well by setting $Q_3^2 = Q_1^2$ and $Q_4^2 = Q_2^2$.

\subsection{$F_2^\gamma$}

We consider first the case of the scattering between a virtual photon and an on-shell photon. In analogy with deep inelastic $e^{\pm} p$ scattering we can think of this process as deep inelastic $e^{\pm} \gamma$ scattering. Just as in the case of deep inelastic $e^{\pm} p$ scattering we can define a hadronic tensor $W^{\mu \nu}$ and two structure functions related by
\begin{align}
\frac{W^{\mu \nu}\left(x, Q^2\right)}{8 \pi^2 \alpha} = &- \left( g^{\mu \nu} + \frac{q_1^\mu q_1^\nu}{Q^2}\right) F_1^\gamma\left(x, Q^2\right) + \notag \\
& + \frac{1}{q_1 \cdot q_2} \left( q_2^\mu + q_1^\mu \frac{q_1 \cdot q_2}{Q^2} \right) \left( q_2^\nu + q_1^\nu \frac{q_1 \cdot q_2}{Q^2} \right) F_2^\gamma \left(x, Q^2\right)
\end{align}
and the cross section for the process $e \gamma \rightarrow e X$ can be written as
\begin{equation}
\frac{d^2 \sigma}{dx dy} = \frac{4 \pi \alpha ^2}{x y Q^2} \left[ \left(1-y\right) F_2^\gamma\left(x, Q^2\right) + x y^2 F^\gamma_1 \right]
\end{equation}
The structure functions are related to the total cross-sections of equation (\ref{eq:ee_eeX_sigma}) through the relations
\begin{align}
	\label{eq:F2_def}
	&F_2^{\gamma} \left(x, Q^2\right) = \frac{Q^2}{4 \pi^2 \alpha} \left[ \sigma_{TT}\left(s, Q^2, Q_2^2 = 0\right) +   \sigma_{LT}\left(s, Q^2, Q_2^2 = 0\right)\right] , \\
	&2 xF_1^{\gamma} \left(x, Q^2\right) = \frac{Q^2}{4 \pi^2 \alpha} \sigma_{TT}\left(s, Q^2, Q_2^2 = 0\right)
\end{align}

Before continuing with the holographic computation let us discuss in which kinematical region will it be applicable. The hadronic photon structure function $F_2^\gamma$ differs from the proton structure function due to the point-like coupling of the photon to the quarks. This coupling makes the photon structure function to rise towards large values of Bjorken $x$ while in the case of the proton it decreases. Moreover $F_2^\gamma$ has positive scaling violations for all values of $x$ while $F_2^p$ has positive scaling violations only at small values of $x$. Also, the point-like part can be evaluated at all orders in perturbative QCD and dominates for large values of $Q^2$ and for values of $x > 0.1$.
On the other hand, the hadronic-like part can not be computed in perturbative QCD. Like its $F_2^p$ counterpart, only its evolution with $Q^2$ can be determined. As in the case of the proton, an ansatz for the $x$ dependence at some scale $Q_0^2$ is given as input to QCD evolution equations. This ansatz can be derived from the Vector Meson Dominance model since the photon can fluctuate in a Vector Meson like the $\rho$ meson. After that they assume that the $F_2^\rho$ structure function is the same as the $F_2^{\pi^0}$ structure function which has been measured experimentally. Then this hadron-like component can be evolved using perturbative QCD and we can compare it with the $F_2^\gamma$ structure function that contains both the point-like and hadron-like contributions. The result is that although the hadron-like component is not important for high values of $Q^2$ and $x > 0.1$ it clearly dominates for very small values of $x$, meaning that we can use the Pomeron exchange picture to study this process. Hence our holographic expression is only valid for $x < 0.01$. We now proceed to the holographic computation of $F_2^\gamma$ in this kinematical region.

As mentioned one the photons in this process is quasi-real. Let us assume that such photon is the represented by the lower part of the Witten diagram. Then the $\bar{z}$ integral simplifies to 
\begin{equation}
\int d\bar{z} e^{- \left(j_n - 3/2\right) A} {\psi_n \left(j_n\left(0\right), \bar{z}\right)}^{*}.
\end{equation}
Defining
\begin{equation}
g_n = - \frac{\pi}{2} \left( i + \cot \frac{\pi j_n\left(0\right)}{2} \right) \frac{k^2_{j_n\left(0\right)}}{2^{j_n\left(0\right)}} j_n'\left(0\right) \int d\bar{z} e^{- \left(j_n - 3/2\right) A} {\psi_n \left(j_n\left(0\right), \bar{z}\right)}^*
\label{eq:gn_def_gammagamma}
\end{equation}
and using the optical theorem we can write the total cross-sections $\sigma_{TT}$ and $\sigma_{LT}$ as
\begin{align}
&\sigma_{TT} =  \sum_n Im\left(g_n\right) s^{j_n - 1} \int dz e^{-\left(j_n - 3/2\right) A} f_Q^2 \psi_n\left(z\right) \\
&\sigma_{LT} =  \sum_n  Im\left(g_n\right) s^{j_n - 1} \int dz e^{-\left(j_n - 3/2\right) A} \frac{\dot{f}_Q^2}{Q^2} \psi_n\left(z\right).
\end{align}
Using the equation (\ref{eq:F2_def}) and $s  = Q^2 / x$ the holographic expression for $F_2^\gamma$ is then
\begin{equation}
F_2^\gamma = \sum_n \frac{ Im \, g_n}{4 \pi^2 \alpha} Q^{2 j_n} x^{1- j_n} \int dz e^{- \left( j_n - 3/2\right) A} \left( f_Q^2 + \frac{\dot{f}_Q}{Q^2} \right) \psi_n \,.
\end{equation}


%%%%%%%%%%%%%%%%%%%%%%%%%%%%%%%%%
\subsection{$\sigma\left(\gamma \gamma \rightarrow X\right)$}
In this process both photons are considered quasi-real, i.e. $Q_1^2 \approx 0$ and $Q_2^2 \approx 0$. In our holographic setup the non-normalizable modes of the bulk $U\left(1\right)$ gauge field
satisfy
\begin{equation}
\lim_{Q \rightarrow 0} f_Q \left(z\right) = 1 \quad , \quad \lim_{Q \rightarrow 0} \frac{\dot{f_Q}}{Q} = 0
\end{equation}
This implies, by using equation (\ref{eq:scattering_amplitude}) and the optical theorem, that~$\sigma^{LL}\left( 0, 0, W\right) = \sigma^{TL}\left(0, 0, W\right) = \sigma^{LT}\left(0, 0, W\right)$ vanish. This is expected since real photons only have transverse polarisation and hence the cross-sections that involve at least one longitudinal on-shell photon do not contribute for this process.
Then 
\begin{equation}
\sigma\left(\gamma \gamma \rightarrow X\right) =  \sigma^{TT}\left(0, 0, s = W^2\right) = \sum_n Im \, g_n s^{j_n - 1} \int dz e^{- \left( j_n - 3/2\right) A}  \psi_n.
\end{equation}
The numbers $Im \, g_n$ have the same definition as the ones in our  holographic expression $F_2^\gamma$ and then these observables are related through the identity
\begin{equation}
\sigma\left(\gamma \gamma \rightarrow X\right) = 4 \pi^2 \alpha \lim_{Q^2 \rightarrow 0} \frac{F_2^\gamma}{Q^2}
\end{equation}
This suggests that if we get good individual fits for $F^\gamma_2$ data and for $\sigma\left(\gamma \gamma \rightarrow X\right)$ data, a joint fit with these two observables should be attempted with the same set of parameters.



%%%%%%%%%%%%%%%%%%%%%%%%%%%%%%%%%%%%%%%
\section{$\gamma^*\, p$ observables}

Like in the case of $\gamma^{*} \gamma$ processes we can study the proton structure functions $F_2$ and $F_L$ and the total cross-section $\sigma\left(\gamma p \rightarrow X\right)$. They are all related to the total cross-sections $\sigma^T_{\gamma^* p}$ and $\sigma^L_{\gamma^{*} p}$ of the inelastic process $\gamma^{*} p \rightarrow X$. Here, as in the previous case, T and L refer to the transverse and longitudinal polarisation of the incoming off-shell photon. At low-$x$ the proton structure functions are given by
\begin{align}
&F_2^p\left(x, Q^2\right) = \frac{Q^2}{4 \pi^2 \alpha} \left( \sigma^T_{\gamma^* p}\left(s, Q^2\right) +  \sigma^L_{\gamma^* p}\left(s, Q^2\right)   \right) \\
&F_L^p\left(x, Q^2\right) = \frac{Q^2}{4 \pi^2 \alpha}  \sigma^L_{\gamma^* p}\left(s, Q^2\right)
\end{align}
and since only on-shell photons have transverse polarisation states, the total cross-section of the process $\gamma p \rightarrow X$ is related to the proton structure function $F_2^p\left(x, Q^2\right)$ through
\begin{equation}
\sigma\left(\gamma p \rightarrow X\right) = 4 \pi^2 \alpha \lim_{Q^2 \rightarrow 0} \frac{F_2\left(x, Q^2\right)}{Q^2}.
\end{equation}

In previous works we have derived holographic expressions for $F_2$ and $F_L$ in the context of the IHQCD model. Hence we will present them below without further justification. For further details the reader can consult the references~\cite{ballon_bayona_unity_2017, Amorim:2018yod, gluonPDF_IHQCD_2020}. 
In IHQCD we have derived the following expressions for the structure functions
\begin{align}
&F_2^p(x, Q^2) = \sum_{n} \frac{ Im g_n}{4 \pi^2 \alpha} Q^{2 j_n} x^{1-j_n} \int dz \,e^{-\left(j_n-\frac{3}{2}\right)A}  \left( f_Q^2  +  \frac{\dot{f}_Q^{2}}{Q^2}      \right) \psi_n , \\
&F_L^p(x, Q^2) = \sum_{n} \frac{Im g_n}{4 \pi^2 \alpha} Q^{2 j_n} x^{1-j_n} \int dz \,e^{-\left(j_n-\frac{3}{2}\right)A}  \frac{\dot{f}_Q^{2}}{Q^2}  \psi_n.
\end{align}
From the expression of $F_2^p$, using $x = Q^2 / s$ and that for $Q = 0, \, f_Q\left(z\right) = 1$ our holographic expression for the total  $\gamma p \rightarrow X$ cross-section is
\begin{equation}
\sigma\left(\gamma p \rightarrow X\right) =  \sum_{n} Im g_n \, s^{j_n -1 } \int dz \,e^{-\left(j_n-\frac{3}{2}\right)A}  \psi_n.
\end{equation}
The definition of $g_n$ for these processes is
\begin{equation}
g_n = - \frac{\pi}{2} \left( i + \cot \frac{\pi j_n}{2} \right) \frac{k_{j_n} \bar{k}_{j_n}}{2^{j_n}} \frac{d j_n}{dt} \int d\bar{z} e^{-\left(j_n - 7/2 \right)A} \upsilon_m^2 {\psi_n}^*
\label{eq:gn_def_gammap}
\end{equation}

\section{$\sigma\left(p p \rightarrow X\right)$}

In this section we compute the total cross-section $\sigma\left(p p \rightarrow X \right)$ in terms of the holographic quantities in our model.
In~\cite{Ballon-Bayona:2015wra} a fit to $p \bar{p}$ total cross-section data was performed by considering the graviton Regge trajectory in IHQCD as a model for the soft pomeron exchange that occurs for high energy scattering processes. The proton and the anti-proton were dual to two different scalar fields in the bulk and their coupling to the spin J fields of the graviton Regge trajectory is of the same form as equation (\ref{eq:scalar_spin_J_coupling}) but with different coupling constants for each scalar field. Since we are considering proton-proton scattering we can use equation (36) of~\cite{Ballon-Bayona:2015wra} and make $\upsilon_2' = \upsilon_1 = \upsilon_p$ and $k_J = k'_J = \bar{k}_J$ to compute the forward scattering amplotude. When we do that we obtain throught the optical theorem
\begin{equation}
\sigma\left(p p \rightarrow X\right) = \sum_n Im g_n \, s^{j_n -1},
\end{equation}
where $g_n$ is defined in this case as
\begin{equation}
g_n = - \frac{\pi}{2} \left( i + \cot \frac{\pi j_n}{2} \right) \frac{\bar{k}^2_{j_n}}{2^{j_n}} \frac{d j_n}{dt}  \int d z d\bar{z} e^{-\left(j_n - 7/2 \right)\left(A+\bar{A}\right)} {|\upsilon_p\left(z\right)|}^2 {|\upsilon_p\left(\bar{z}\right)|}^2 \psi_n\left(z\right){\psi_n}^{*}\left(\bar{z}\right)
\label{eq:pp_fit_constant}
\end{equation}

\section{Data analysis and results}

Comparing the holographic formulas in the previous sections with experiments allows us to determine the constants ${\rm Im} \, g_n$ that describe best the available data. These constants have different definitions for $\gamma^*\gamma$, $\gamma^*p$ and $pp$ processes, so we will determine a set of values ${\rm Im} \, g_n$ for each process class. We will use the first four Reggeons, which were enough to describe the proton structure functions $F_2\left(x, Q^2\right)$ and $F_L\left(x, Q^2\right)$ in~\cite{gluonPDF_IHQCD_2020}. Each ${\rm Im} \, g_n$ is associated with a Reggeon. The best set of parameters is obtained by performing a weighted $\chi^2$ fit to a dataset. For the structure functions of the proton and photon the weights are the inverse of the experimental error of each point. For total cross-section data we also need to take into account that some data points have uncertainties in the values of $s$ (e.g. in $\gamma \gamma \rightarrow X$ that is always the case because it is a measured quantity). To account for this we compute the total cross-section for $s + \Delta s$ and $s - \Delta s$ and compute
\begin{equation}
\Delta \sigma_{\rm eff.} = {\rm max} \left(|\sigma^{\rm pred}\left(s + \Delta s\right)-\sigma^{\rm pred}\left(s\right)|,|\sigma^{\rm pred}\left(s - \Delta s\right)-\sigma^{\rm pred}\left(s\right)| \right)
\end{equation}
The weight for these cases is the inverse of ${\left(\Delta \sigma_{\rm exp.}\right)}^2+{\left(\Delta \sigma_{\rm eff.}\right)} ^2$ where $\Delta \sigma_{\rm exp.}$ is the experimental error.

For $\sigma\left(\gamma \gamma \rightarrow X\right)$,  $\sigma\left(\gamma p \rightarrow X\right)$ and $\sigma\left(p p \rightarrow X\right)$  fits we use the hadronic cross-section data files from Particle Data Group~\cite{pdg_2018}. These data sets are a compilation of experimental results obtained in the last decades from several collaborations. The datasets of  $\sigma\left(\gamma p \rightarrow X\right)$ and $\sigma\left(p p \rightarrow X\right)$ had cross-section values as a function of the laboratory momentum of the incoming on-shell photon and proton respectively. Hence we computed the respective center of mass energy $\sqrt{s}$ before performing the fits. For the fits involving total cross-sections we also considered only subsets with $\sqrt{s} > 4$, $\sqrt{s} > 4.6$ and $\sqrt{s} > 3 \, \text{GeV}$ for $\sigma\left(\gamma \gamma \rightarrow X\right)$, $\sigma\left(\gamma p \rightarrow X\right)$ and $\sigma\left(p p \rightarrow X \right)$ respectively. This gives us for $\sigma\left(\gamma \gamma \rightarrow X\right)$, $\sigma\left(\gamma p \rightarrow X\right)$ and $\sigma\left(p p \rightarrow X \right)$ 39, 45 and 150 data points, respectively.  The lower bound cuts result from the fact that our model does not realise the meson trajectory with an intercept around $0.35-0.55$. This trajectory dominates for smaller values of $s$ and the best our model can do is to mimic it in the intermediate range of $s$ through the third and fourth trajectories. In the future we plan to study how to include the meson trajectory and redo these fits.

The source of data for the proton structure functions $F_2\left(x, Q^2\right)$ and $F_L\left(x, Q^2\right)$ we use the HERA measurements of these observables available in~\cite{Aaron:2009aa, Collaboration:2010ry}. There are 249 points in the kinematical region with $x < 10^{-2}$ and $Q^2 \leq 400 {\rm \, GeV}^2$ for $F_2$ and 64 points with $x < 10^{-2}$ and $Q^2 \leq 45 {\rm \, GeV}^2$ for $F_L$. For the photon structure function $F_2^\gamma\left(x, Q^2\right)$ we consider the measurements of ALEPH~\cite{Heister:2003an}, L3~\cite{Acciarri:1998ig} and OPAL~\cite{Ackerstaff:1997ng, Abbiendi:2000cw, Abbiendi:2002te} collaborations at LEP and of the TPC/Two Gamma collaboration ~\cite{Aihara:1986xw} at SLAC $e^+e^-$ storage ring PEP. These measurements contribute with 22 points with $x \leq 0.0235$ and $Q^2 \leq 17.8 {\rm GeV}^2$.

The minimisation of the $\chi^2$ is done using an implementation in C++ of the Nelder-Mead (NM) algorithm presented in the book Numerical Recipes. We are performing fits with 4 free parameters and hence it is hard to tell if the minimum found by the NM algorithm is a local or global minimum. To solve this issue we tried several initial guesses in the parameter space and the values we quote are the lowest value for the $\chi^2$ per number of degrees of freedom (i.e. ${\rm N_{points} - N_{parameters}}$) we have found.

\begin{table}[b!]
\centering
\caption{Values of the parameters for the fit with $\sigma\left(\gamma \gamma \rightarrow X\right)$ data given by PDG with $\sqrt{s} > 4 {\rm \, GeV}$ and using the pomeron kernel of~\cite{gluonPDF_IHQCD_2020}. The total number of data points is 39 and a $\chi^2 / {N.d.o.f.}$ of 0.593086 was obtained.}
\vspace{0.5cm}
\begin{tabular}{|c|c|}
\hline
couplings   & value \\
\hline
${\rm Im} \, g_0$  & 0.000805106 \\ 
\hline
${\rm Im} \, g_1$  & 0.00269262 \\ 
\hline
${\rm Im} \, g_2$  & 0.00678736 \\ 
\hline
${\rm Im} \, g_3$  & -0.00216957 \\ 
\hline
\end{tabular}
\label{table:SigmaGammaGamma_best_fit_PDG_W_gt_4_HardPomeron}
\end{table}

\begin{figure}[!b]
\center
\includegraphics[scale = 0.5]{images/SigmaGammaGamma_vs_W_HardPomeron_PDG_data_only_W_gt_4.pdf} 
\caption{Predicted $\sigma\left(\gamma \gamma \rightarrow X\right)$ vs experimental points. Our curve was obtained using the parameter values in table ~\ref{table:SigmaGammaGamma_best_fit_PDG_W_gt_4_HardPomeron}.}
\label{fig:SigmaGammaGamma_best_fit_Sigma_data_only}
\end{figure}

\begin{table}[b!]
\centering
\caption{Values of the parameters for the fit with $F_2^{\gamma}\left(x, Q^2\right)$ data only and using the pomeron kernel of~\cite{gluonPDF_IHQCD_2020}. The total number of data points is 22 and a $\chi^2 / {N.d.o.f.}$ of 1.00065 was obtained.}
\vspace{0.5cm}
\begin{tabular}{|c|c|}
\hline
couplings   & value \\
\hline
${\rm Im} \, g_0$  & 8.51615e-05\\ 
\hline
${\rm Im} \, g_1$  & 0.00071289 \\ 
\hline
${\rm Im} \, g_2$  & 0.000483367  \\
\hline
${\rm Im} \, g_3$  & -0.00316919\\ 
\hline
\end{tabular}
\label{table:F2Photon_best_fit}
\end{table}

\begin{figure}[!b]
\center
\includegraphics[scale = 0.5]{images/F2Photon_F2Photon_data_only.pdf} 
\caption{Predicted $F_2^\gamma\left(x, Q^2\right)$ vs experimental points. Our curve was obtained using the values from table ~\ref{table:F2Photon_best_fit}.}
\label{fig:F2Photon_best_fit_F2_data_only}
\end{figure}

For the $\gamma^* \gamma$ processes we started to do individual fits of $\sigma\left(\gamma \gamma \rightarrow X\right)$ and $F_2^\gamma$ data. We have found a $\chi^2$ around 0.6 and 1 for $\sigma\left(\gamma \gamma \rightarrow X\right)$ and $F_2^\gamma$ respectively. The corresponding best fit parameters can be found in tables~\ref{table:SigmaGammaGamma_best_fit_PDG_W_gt_4_HardPomeron} and~\ref{table:F2Photon_best_fit} and the comparison of the best fits with data can be seen in figures~\ref{fig:SigmaGammaGamma_best_fit_Sigma_data_only} and~\ref{fig:F2Photon_best_fit_F2_data_only}. 
\begin{table}[b!]
\centering
\caption{Values of the parameters for the joint fit of the photon structure function $F_2^{\gamma}\left(x, Q^2\right)$ and $\sigma\left(\gamma \gamma \rightarrow X\right)$ data from PDG with $\sqrt{s} > 4 {\rm \, GeV}$. The total number of data points is 61 and a $\chi^2 / {N.d.o.f.}$ of 1.05567 was obtained.}
\vspace{0.5cm}
\begin{tabular}{|c|c|}
\hline
couplings   & value \\
\hline
${\rm Im} \, g_0$  & 0.000176017\\ 
\hline
${\rm Im} \, g_1$  & 0.00022995 \\ 
\hline
${\rm Im} \, g_2$  & -0.000173646 \\
\hline
${\rm Im} \, g_3$  & -0.001411 \\ 
\hline
\end{tabular}
\label{table:GammaGamma_best_joint_fit}
\end{table}
\begin{figure}[!b]
\center
\includegraphics[scale = 0.5]{images/SigmaGammaGamma_vs_W_joint_fit_PDG_W_gt_4.pdf} 
\caption{Predicted $\sigma\left(\gamma \gamma \rightarrow X\right)$ vs experimental points. Our curve was obtained using the parameter values in table ~\ref{table:GammaGamma_best_joint_fit}.}
\label{fig:SigmaGammaGamma_best_joint_fit}
\end{figure}
\begin{figure}[!b]
\center
\includegraphics[scale = 0.5]{images/F2Photon_PDG_W_gt_4.pdf} 
\caption{Predicted $F_2^\gamma\left(x, Q^2\right)$ vs experimental points. Our curve was obtained using the values from table ~\ref{table:GammaGamma_best_joint_fit}.}
\label{fig:F2Photon_best_joint_fit}
\end{figure}
Because the definition of the fitting parameters is the same for both processes and we have got individual good fits a joint fit was attempted with a total of  61 experimental points. A $\chi^2$ around $1.06$ was obtained and the best fit parameters are displayed in table~\ref{table:GammaGamma_best_joint_fit}.

\begin{table}[b!]
\centering
\caption{Values of the parameters for the best fit of $\sigma\left(\gamma p \rightarrow X\right)$ using only data from PDG with $\sqrt{s} > 4.6 \, \text{GeV}$ and the pomeron kernel of~\cite{gluonPDF_IHQCD_2020}. A total number of 45 points has been used resulting in a~$\chi^2 / {N.d.o.f.}$ of 0.762536. }
\vspace{0.5cm}
\begin{tabular}{|c|c|}
\hline
couplings   & value \\
\hline
${\rm Im} \, g_0$  & 0.000410538\\ 
\hline
${\rm Im} \, g_1$  & -0.392161 \\ 
\hline
${\rm Im} \, g_2$  & -6.19442  \\
\hline
${\rm Im} \, g_3$  & -1.53268\\ 
\hline
\end{tabular}
\label{table:SigmaGammaP_HardPomeron_PDG_best_fit}
\end{table}
\begin{figure}[!b]
\center
\includegraphics[scale = 0.5]{images/SigmaGammaP_vs_W_PDG_data_only_W_gt_461.pdf} 
\caption{Predicted $\sigma \left( \gamma p \rightarrow X \right)$ vs experimental points. Our curve was obtained using the values from table ~\ref{table:SigmaGammaP_HardPomeron_PDG_best_fit}.}
\label{fig:SigmaGammaProton_data_only_best_fit}
\end{figure}
For $\gamma^{*}p$ processes we started first with an individual fit of $\sigma\left(\gamma p \rightarrow X\right)$. We got a $\chi^2$ of $0.76$ with the parameters of table~\ref{table:SigmaGammaP_HardPomeron_PDG_best_fit} and our model prediction against the experimental points is displayed in figure~\ref{fig:SigmaGammaProton_data_only_best_fit}. The fit results using the proton structure functions $F_2$ and $F_L$ are present in~\cite{gluonPDF_IHQCD_2020} and a $\chi^2$ around $1.4$ was obtained. For the joint fit of the proton structure functions with $\sigma\left(\gamma p \rightarrow X\right)$, with a total of 358 points, we got also a $\chi^2$ of $1.4$ with the parameter values of table~\ref{table:joint_DIS_GammaP_best_joint_fit_PDG}. Our predictions against the $\sigma\left(\gamma p \rightarrow X\right)$ data are present in figure~\ref{fig:SigmaGammaProton_best_joint_fit} and agains the proton structure function $F_2$ and $F_L$ data are in figures~\ref{fig:F2_best_joint_fit} and~\ref{fig:FL_best_joint_fit}.
\begin{table}[b!]
\centering
\caption{Values of the parameters for the best joint fit of the proton structure functions $F_2\left(x, Q^2\right)$ and $F_L\left(x, Q^2\right)$ and $\sigma\left(\gamma p \rightarrow X\right)$ data from PDG with $x \leq 0.01$, $Q^2 \leq 400 \, \text{GeV}^2$ and $\sqrt{s} > 4.6 \, \text{GeV}$. This means a~$\chi^2 / {N.d.o.f.}$ of 1.43779 with 358 experimental points.}
\vspace{0.5cm}
\begin{tabular}{|c|c|}
\hline
couplings   & value \\
\hline
${\rm Im} \, g_0$  & 0.0494111\\ 
\hline
${\rm Im} \, g_1$  & 0.0226401 \\ 
\hline
${\rm Im} \, g_2$  & -0.000212687  \\
\hline
${\rm Im} \, g_3$  & -0.361222\\ 
\hline
\end{tabular}
\label{table:joint_DIS_GammaP_best_joint_fit_PDG}
\end{table}
\begin{figure}[!b]
\center
\includegraphics[scale = 0.5]{images/SigmaGammaP_vs_W_PDG_data_W_gt_461.pdf} 
\caption{Predicted $\sigma \left( \gamma p \rightarrow X \right)$ vs experimental points. Our curve was obtained with our model with the values of table~\ref{table:joint_DIS_GammaP_best_joint_fit_PDG}.}
\label{fig:SigmaGammaProton_best_joint_fit}
\end{figure}
\begin{figure}[!b]
\center
\includegraphics[scale = 0.5]{images/Holographic_F2_splitted_joint_fit_PDG_1.pdf} 
\caption{Predicted proton structure function $F_2\left(x,Q^2\right)$ vs experimental points. Our curve was obtained using our model with the values of table~\ref{table:joint_DIS_GammaP_best_joint_fit_PDG}.}
\label{fig:F2_best_joint_fit}
\end{figure}
\begin{figure}[!b]
\center
\includegraphics[scale = 0.5]{images/Holographic_FL_splitted_joint_fit_PDG_1.pdf} 
\caption{Predicted proton structure function $F_L\left(x,Q^2\right)$ vs experimental points. Our curve was obtained using our model with the values of table~\ref{table:joint_DIS_GammaP_best_joint_fit_PDG}.}
\label{fig:FL_best_joint_fit}
\end{figure}


\begin{table}[b!]
\centering
\caption{Values of the parameters for $\sigma\left(p p \rightarrow X\right)$ data with $4.54 < \sqrt{s} \leq 386 \text{GeV}$. This results in a $\chi^2 / {N.d.o.f.}$ of 1.05721 with 113 experimental points.}
\vspace{0.5cm}
\begin{tabular}{|c|c|}
\hline
couplings   & value \\
\hline
$g_0$  & 33.4557\\ 
\hline
$g_1$  & -64.087 \\ 
\hline
$g_2$  & 91.6046  \\
\hline
$g_3$  & 56.0437\\ 
\hline
\end{tabular}
\label{table:SigmaPP_best_fit_W_lt_386}
\end{table}

\begin{figure}[!h]
\center
\includegraphics[width = \textwidth]{images/SigmaProtonProton_HardPomeron_W_lt_386.pdf} 
\caption{Predicted $\sigma\left(p p \rightarrow X\right)$ vs experimental points. Our curve was obtained using the values from table ~\ref{table:SigmaPP_best_fit_W_lt_386}.}
\label{fig:SigmaPP_best_fit_W_lt_386}
\end{figure}

Finally we present our best fit results for $\sigma\left(p p \rightarrow X\right)$. We got a $\chi^2$ of 0.9 for this process using the values of table~\ref{table:SigmaPP_best_fit}. For those values the predicted dependence of the total cross-section of proton-proton scattering is the one appearing in figure~\ref{fig:SigmaPP_best_fit}. Recall that the first parameter is associated with the hard pomeron while the second parameter is associated with the soft pomeron. Having said this we can tell that for high center of mass energies the hard pomeron's contribution is small as compared to the soft-pomeron one. This is consistent with the claims that the hard-pomeron is not important to describe hadron-hadron total cross-section for center of mass energies below $1 {\rm \, TeV}$. In~\cite{Donnachie:2011aa} Donnachie and Landshoff have argued that a moderate hard-pomeron contribution is also present in hadron-hadron scattering in order to describe total cross-section at center of mass energies above $1 {\rm \, TeV}$. In their work the hard-pomeron intercept is of $1.362$.

\begin{table}[b!]
\centering
\caption{Values of the parameters for $\sigma\left(p p \rightarrow X\right)$ data with $ \sqrt{s} > 3 \text{GeV}$. This results in a $\chi^2 / {N.d.o.f.}$ of 0.881516 with 150 experimental points.}
\vspace{0.5cm}
\begin{tabular}{|c|c|}
\hline
couplings   & value \\
\hline
${\rm Im} \, g_0$  & -7.47707\\ 
\hline
${\rm Im} \, g_1$  & 128.449 \\ 
\hline
${\rm Im} \, g_2$  & -361.212  \\
\hline
${\rm Im} \, g_3$  & 370.23 \\ 
\hline
\end{tabular}
\label{table:SigmaPP_best_fit}
\end{table}

\begin{figure}[!h]
\center
\includegraphics[scale = 0.5]{images/SigmaProtonProton_HardPomeron.pdf} 
\caption{Predicted $\sigma\left(p p \rightarrow X\right)$ vs experimental points. Our curve was obtained using the values from table ~\ref{table:SigmaPP_best_fit}.}
\label{fig:SigmaPP_best_fit}
\end{figure}

In the Mathematica notebook we realized that after computing the $\bar{z}$ integral for the $\gamma^{*} \gamma$ processes and using the best fit values of ${\rm Im} \, g_n$ one would obtain pure imaginary gravitational couplings $k_{j_1}$ and $k_{j_4}$. Our model predictions are symmetric under $\psi_n \rightarrow e^{i\phi} \psi_n$ because of the product $\psi_n {\psi_n}^*$. Hence we rescaled the wavefunctions for $n=1$ and $n=4$. After we have done that we repeated the joint fit of $F_2^\gamma$ and $\sigma\left(\gamma \gamma \rightarrow X\right)$. We obtained again a $\chi^2$ around 1.06 with the parameter values of table~\ref{table:GammaGamma_best_joint_fit_scaled}. As expected the numerical values are very close to the ones of table~\ref{table:GammaGamma_best_joint_fit} except the minus sign of ${\rm Im} \, g_1$ and ${\rm Im} \, g_4$. 
\begin{table}[b!]
\centering
\caption{Values of the parameters for the joint fit of the photon structure function $F_2^{\gamma}\left(x, Q^2\right)$ and $\sigma\left(\gamma \gamma \rightarrow X\right)$ data from PDG with $\sqrt{s} > 4 {\rm \, GeV}$. The total number of data points is 61 and a $\chi^2 / {N.d.o.f.}$ of 1.0556 was obtained.}
\vspace{0.5cm}
\begin{tabular}{|c|c|}
\hline
couplings   & value \\
\hline
${\rm Im} \, g_0$  & -0.000176569\\ 
\hline
${\rm Im} \, g_1$  & 0.000228931 \\ 
\hline
${\rm Im} \, g_2$  & -0.000184843 \\
\hline
${\rm Im} \, g_3$  & 0.00140601 \\ 
\hline
\end{tabular}
\label{table:GammaGamma_best_joint_fit_scaled}
\end{table}
I repeated the fit for the $\gamma^{*}p$ processes and got again a $\chi^2$ of 1.4 for the joint fit involving the proton structure functions $F_2$ and $F_L$ and $\sigma\left(\gamma^* p \rightarrow X\right)$. The parameters found are the ones present in table~\ref{table:GammaProton_best_joint_fit_scaled} and, as expected, are also close to the values in table~\ref{table:joint_DIS_GammaP_best_joint_fit_PDG} except to the sign difference in ${\rm Im} \, g_1$ and ${\rm Im} \, g_4$ due to the wavefunction rescaling.
\begin{table}[b!]
\centering
\caption{Values of the parameters for the best joint fit of the proton structure functions $F_2\left(x, Q^2\right)$ and $F_L\left(x, Q^2\right)$ and $\sigma\left(\gamma p \rightarrow X\right)$ data from PDG with $x \leq 0.01$, $Q^2 \leq 400 \, \text{GeV}^2$ and $\sqrt{s} > 4.6 \, \text{GeV}$. This means a~$\chi^2 / {N.d.o.f.}$ of 1.4376 with 358 experimental points. The wavefunctions have been scaled.}
\vspace{0.5cm}
\begin{tabular}{|c|c|}
\hline
couplings   & value \\
\hline
${\rm Im} \, g_0$  & -0.0495025\\ 
\hline
${\rm Im} \, g_1$  & 0.0223465 \\ 
\hline
${\rm Im} \, g_2$  & -0.000909831 \\
\hline
${\rm Im} \, g_3$  & 0.360583 \\ 
\hline
\end{tabular}
\label{table:GammaProton_best_joint_fit_scaled}
\end{table}

The definitions of $\rm{Im} \, g_n$ of $\gamma^* \gamma$ and of $\rm{Im} \, g_n$ of $\gamma^* p$ processes suggest that a joint fit of $F_2^\gamma$, $\sigma\left(\gamma \gamma \rightarrow X\right)$, $F_2$, $F_L$ and $\sigma\left(\gamma p \rightarrow X\right)$ should be possible. The fitting parameters in this fit are the gravitational couplings $k_{j_n}$ between the bulk $U\left(1\right)$ gauge field with the n-th reggeon and the product of the gravitational couplings $\bar{k}_{j_n}$ of the bulk field dual to the proton with the $\bar{z}$ integral.
We have done such a fit making use of the combined 419 experimental points of the different observables. A $\chi^2$ of $1.38$ was obtained with the parameters of table~\ref{table:GammaGamma_GammaProton_best_joint_fit}.
\begin{table}[b!]
\centering
\caption{Best fit values for the joint fit of $\gamma^*\gamma$ and $\gamma^*p$ processes. 419 experimental points were used resulting in a $\chi^2$ of $1.38$.}
\begin{tabular}{|c|c|c|c|}
\hline
$k_{j_n}$ & value & $\bar{k}_{j_n} \times \rm{\bar{z} \, integral \, value}$ & value \\
\hline
$n = 1$ & 0.0480899 & n = 1 & -16.8191 \\
\hline
$n = 2$ & 0.0707613 & n = 2 & -6.26249 \\
\hline
$n = 3$ & 0.167822 & n = 3 & 0.126023 \\
\hline
$n = 4$ & 0.264574 & n = 4 & 52.6098 \\
\hline
\end{tabular}
\label{table:GammaGamma_GammaProton_best_joint_fit}
\end{table}

This last fit allowed us to determine the product of gravitational couplings $\bar{k}_{j_n}$ of the bulk field dual to the proton with the $\bar{z}$ integral containing the wave function of the proton in the bulk. Then using equation (\ref{eq:pp_fit_constant}) one obtains
\begin{align}
& {\rm Im \, g_1} = -17.3092 \, , & {\rm Im \, g_2} = -1.97037 \notag \\
& {\rm Im \, g_3} = -0.000535407 \, , & {\rm Im \, g_4} =	-71.5961
\end{align}
Comparing these values with the ones obtained in the fit using only $\sigma\left(p p \rightarrow X\right)$ data we see that they are off by orders of magnitude. Moreover since the $\bar{k}_{j_n}$ are real numbers we can see from equation (\ref{eq:pp_fit_constant}) that ${\rm Im \, g_n}$ should be negative for all n. However, the fit using only $\sigma\left(p p \rightarrow X\right)$ data demanded that some of the constants to be positive.

\section{Fits with cuts on $\sqrt{s}$}

We noticed that  sometimes the best fit parameters of a joint fit are very different from fits using only data of one observable. This is seen both in $\gamma^{*} \gamma$ and $\gamma^{*} p$ processes. We believe that this is due to the fact that we try to apply our model in a region of low $\sqrt{s}$. This is the region where the meson trajectory, which has an intercept lower than 1, dominates. Our model besides the hard pomeron and soft pomeron trajectories has two more Regge trajectories with intercepts smaller than 1. However it does not include a trajectory dual to the meson trajectory. Then it is possible that that the two trajectories, for a given set of parameters, can mimic the meson trajectory in the low $\sqrt{s}$ region. Since these two trajectories are also important to describe the photon and proton structure functions it may happen that forcing our model to fit $\sigma$ data in this region may imply a poorer description of these functions. 

Considering a fit using only $\sigma\left(\gamma p \rightarrow X\right)$ data with $\sqrt{s} > 10 \, \text{GeV}$ we obtained the best fit parameters present in table~\ref{table:SigmaGammaProton_W_gt_10_best_fit_scaled}. The parameters on that table are similar numerical values with the ones of table~\ref{table:GammaProton_best_joint_fit_scaled} except for the parameter $\rm{Im} \, g_2$. due to the experimental uncertainties there are uncertainties associated with the best fit parameters and it ma happen that the uncertainties are bigger than the parameter value, making them compatible. Considering fits using only $\sigma\left(\gamma \gamma \rightarrow X\right)$ data with $\sqrt{s} > 20 \, \text{GeV}$ we obtained the best fit parameters present in table~\ref{table:SigmaGammaGamma_W_gt_20_best_fit_scaled}. Again we verify the same pattern as in the previous case: excluding more low $\sqrt{s}$ points made the order of magnitude of the parameters to be the same or almost the same as the ones obtained in the fit using only $F_2^\gamma$ data. Notice that the parameters 1 and 4 of table~\ref{table:F2Photon_best_fit} need to be multiplied by $-1$ since in these new fits we have made $\psi_1\rightarrow -\psi_1$ and $\psi_4\rightarrow -\psi_4$.

\begin{table}[b!]
\centering
\caption{Values of the parameters for the best fit of $\sigma\left(\gamma p \rightarrow X\right)$ data from PDG with $\sqrt{s} > 10 \, \text{GeV}$. This means a ~$\chi^2 / {N.d.o.f.}$ of 0.693373 with 25 experimental points. The wavefunctions have been scaled.}
\vspace{0.5cm}
\begin{tabular}{|c|c|}
\hline
couplings   & value \\
\hline
${\rm Im} \, g_0$  & -0.0531552\\ 
\hline
${\rm Im} \, g_1$  & 0.0332578 \\ 
\hline
${\rm Im} \, g_2$  & -0.0254186 \\
\hline
${\rm Im} \, g_3$  & 0.373644 \\ 
\hline
\end{tabular}
\label{table:SigmaGammaProton_W_gt_10_best_fit_scaled}
\end{table}

\begin{table}[b!]
\centering
\caption{Values of the parameters for the best fit of $\sigma\left(\gamma \gamma \rightarrow X\right)$ data from PDG with $\sqrt{s} > 20 \, \text{GeV}$. This means a ~$\chi^2 / {N.d.o.f.}$ of 0.512296 with 15 experimental points. The wavefunctions have been scaled.}
\vspace{0.5cm}
\begin{tabular}{|c|c|}
\hline
couplings   & value \\
\hline
${\rm Im} \, g_0$  & -0.0003603\\ 
\hline
${\rm Im} \, g_1$  & 0.000655799 \\ 
\hline
${\rm Im} \, g_2$  & -0.000343226 \\
\hline
${\rm Im} \, g_3$  & 0.00125003 \\ 
\hline
\end{tabular}
\label{table:SigmaGammaGamma_W_gt_20_best_fit_scaled}
\end{table}
With these cuts in the center-of-mass energy we can also do new joint fits to $\gamma^*\gamma$ and $\gamma^*p$ processes. For a cut of $\sqrt{s} > 10 \, {\rm GeV}$ for $\sigma\left(\gamma \gamma \rightarrow X \right)$ we get the best fit parameters of $\gamma^* \gamma$ processes of table~\ref{table:GammaGamma_best_joint_fit_scaled_W_gt_10} while for a cut of  $\sqrt{s} > 20 \, {\rm GeV}$ we get the ones of table~\ref{table:GammaGamma_best_joint_fit_scaled_W_gt_20}. For the $\gamma^* p$ processes with the cut of $\sqrt{s} > 10 \, {\rm GeV}$ we get the values of table~\ref{table:GammaProton_best_joint_fit_scaled_W_gt_10}. For the joint fit of $\gamma^* \gamma$ and $\gamma^* p$ processes they are displayed in tables~\ref{table:GammaGamma_GammaProton_best_joint_fit_W_gt_10} and~\ref{table:GammaGamma_GammaProton_best_joint_fit_W_gt_20}.
\begin{table}[b!]
\centering
\caption{Values of the parameters for the joint fit of the photon structure function $F_2^{\gamma}\left(x, Q^2\right)$ and $\sigma\left(\gamma \gamma \rightarrow X\right)$ data from PDG with $\sqrt{s} > 10 {\rm \, GeV}$. The total number of data points is 46 and a $\chi^2 / {N.d.o.f.}$ of 1.15531 was obtained.}
\vspace{0.5cm}
\begin{tabular}{|c|c|}
\hline
couplings   & value \\
\hline
${\rm Im} \, g_0$  & -0.000175226\\ 
\hline
${\rm Im} \, g_1$  & 0.000230161 \\ 
\hline
${\rm Im} \, g_2$  & -0.000154943 \\
\hline
${\rm Im} \, g_3$  & 0.0014227 \\ 
\hline
\end{tabular}
\label{table:GammaGamma_best_joint_fit_scaled_W_gt_10}
\end{table}
\begin{table}[b!]
\centering
\caption{Values of the parameters for the joint fit of the photon structure function $F_2^{\gamma}\left(x, Q^2\right)$ and $\sigma\left(\gamma \gamma \rightarrow X\right)$ data from PDG with $\sqrt{s} > 20 {\rm \, GeV}$. The total number of data points is 37 and a $\chi^2 / {N.d.o.f.}$ of 1.0556 was obtained.}
\vspace{0.5cm}
\begin{tabular}{|c|c|}
\hline
couplings   & value \\
\hline
${\rm Im} \, g_0$  & -0.000163052\\ 
\hline
${\rm Im} \, g_1$  & 0.000273758 \\ 
\hline
${\rm Im} \, g_2$  & 6.51405e-05 \\
\hline
${\rm Im} \, g_3$  & 0.00168805 \\ 
\hline
\end{tabular}
\label{table:GammaGamma_best_joint_fit_scaled_W_gt_20}
\end{table}
\begin{table}[b!]
\centering
\caption{Values of the parameters for the best joint fit of the proton structure functions $F_2\left(x, Q^2\right)$ and $F_L\left(x, Q^2\right)$ and $\sigma\left(\gamma p \rightarrow X\right)$ data from PDG with $x \leq 0.01$, $Q^2 \leq 400 \, \text{GeV}^2$ and $\sqrt{s} > 10 \, \text{GeV}$. This means a~$\chi^2 / {N.d.o.f.}$ of 1.39152 with 338 experimental points. The wavefunctions have been scaled.}
\vspace{0.5cm}
\begin{tabular}{|c|c|}
\hline
couplings   & value \\
\hline
${\rm Im} \, g_0$  & -0.0498758\\ 
\hline
${\rm Im} \, g_1$  & 0.0199297 \\ 
\hline
${\rm Im} \, g_2$  & -0.00480236 \\
\hline
${\rm Im} \, g_3$  & 0.349284 \\ 
\hline
\end{tabular}
\label{table:GammaProton_best_joint_fit_scaled_W_gt_10}
\end{table}
\begin{table}[b!]
\centering
\caption{Best fit values for the joint fit of $\gamma^*\gamma$ and $\gamma^*p$ processes with $\sqrt{s} > 10 \, {\rm GeV }$ cuts on the total cross section data. 384 experimental points were used resulting in a $\chi^2$ of $1.37$.}
\begin{tabular}{|c|c|c|c|}
\hline
$k_{j_n}$ & value & $\bar{k}_{j_n} \times \rm{\bar{z} \, integral \, value}$ & value \\
\hline
$n = 1$ & 0.0470443 & n = 1 & 62.767 \\
\hline
$n = 2$ & -0.0744605 & n = 2 & -40.5151 \\
\hline
$n = 3$ & 0.0720959 & n = 3 & 48.1009 \\
\hline
$n = 4$ & 0.275433 & n = 4 & 355.92 \\
\hline
\end{tabular}
\label{table:GammaGamma_GammaProton_best_joint_fit_W_gt_10}
\end{table}
\begin{table}[b!]
\centering
\caption{Best fit values for the joint fit of $\gamma^*\gamma$ and $\gamma^*p$ processes with $\sqrt{s} > 10 \, {\rm GeV }$ cut for $\sigma\left(\gamma p \rightarrow X\right)$ and $\sqrt{s} > 20 \, {\rm GeV }$ cut for $\sigma\left(\gamma \gamma \rightarrow X\right)$. 375 experimental points were used resulting in a $\chi^2$ of $1.38$.}
\begin{tabular}{|c|c|c|c|}
\hline
$k_{j_n}$ & value & $\bar{k}_{j_n} \times \rm{\bar{z} \, integral \, value}$ & value \\
\hline
$n = 1$ & 0.0469159 & n = 1 & 62.5756 \\
\hline
$n = 2$ & -0.0749142 & n = 2 & -41.4804 \\
\hline
$n = 3$ & -0.0726996 & n = 3 & -2.37336\\
\hline
$n = 4$ & 0.282108 & n = 4 & 350.93 \\
\hline
\end{tabular}
\label{table:GammaGamma_GammaProton_best_joint_fit_W_gt_20}
\end{table}
A fit considering total cross-section data with $10 < \sqrt{s} < 10000 \, \rm{GeV}$ for $pp$ scattering was also considered. The results are shown in table~\ref{table:SigmaProtonProton_best_fit_10_lt_W_lt_10000}. Considering only data with $10 < \sqrt{s} < 386 \, \rm{GeV}$ we have the results in table~\ref{table:SigmaProtonProton_best_fit_10_lt_W_lt_386}.
\begin{table}[b!]
\centering
\caption{Values of the parameters for the best fit of  $\sigma\left(p p \rightarrow X\right)$ data from PDG with $10 < \sqrt{s} < 10000 \, \text{GeV}$. This means a~$\chi^2 / {N.d.o.f.}$ of 1.39786 with 77 experimental points.}
\vspace{0.5cm}
\begin{tabular}{|c|c|}
\hline
couplings   & value \\
\hline
${\rm Im} \, g_0$  & 7.19131 \\ 
\hline
${\rm Im} \, g_1$  & 17.0716 \\ 
\hline
${\rm Im} \, g_2$  & 35.6313 \\
\hline
${\rm Im} \, g_3$  & 36.8793 \\ 
\hline
\end{tabular}
\label{table:SigmaProtonProton_best_fit_10_lt_W_lt_10000}
\end{table}
\begin{table}[b!]
\centering
\caption{Values of the parameters for the best fit of  $\sigma\left(p p \rightarrow X\right)$ data from PDG with $10 < \sqrt{s} < 386 \, \text{GeV}$. This means a~$\chi^2 / {N.d.o.f.}$ of 1.22972 with 71 experimental points.}
\vspace{0.5cm}
\begin{tabular}{|c|c|}
\hline
couplings   & value \\
\hline
${\rm Im} \, g_0$  & 9.62334 \\ 
\hline
${\rm Im} \, g_1$  & 13.1395 \\ 
\hline
${\rm Im} \, g_2$  & 21.6487 \\
\hline
${\rm Im} \, g_3$  & 56.3791 \\ 
\hline
\end{tabular}
\label{table:SigmaProtonProton_best_fit_10_lt_W_lt_386}
\end{table}

Given the values of the gravitational couplings in tables~\ref{table:GammaGamma_GammaProton_best_joint_fit_W_gt_10} and~\ref{table:GammaGamma_GammaProton_best_joint_fit_W_gt_20} we can compute the ${\rm Im} \, g_n$ of the $\gamma^{*} \gamma$, $\gamma^* p$ and pp processes using the definitions of equations (\ref{eq:gn_def_gammagamma}), (\ref{eq:gn_def_gammap}) and (\ref{eq:pp_fit_constant}). We note by definition the ${\rm Im} \, g_n$ of pp should all be negative while in tables~\ref{table:SigmaProtonProton_best_fit_10_lt_W_lt_10000} and \ref{table:SigmaProtonProton_best_fit_10_lt_W_lt_386} they are all positive. To make all of them positive we make the gravitational couplings of the proton to be redefined by $i \bar{k}_{j_n}$. This means that the definition of the ${\rm Im} \, g_n$ of the $\gamma^* p$ processes picks an extra factor of $\cot \frac{\pi j_n}{2}$. Having said this using the gravitational couplings values of table~\ref{table:GammaGamma_GammaProton_best_joint_fit_W_gt_10} we get the values of table~\ref{table:Im_gn_vals_1} while using the gravitational couplings values of table~\ref{table:GammaGamma_GammaProton_best_joint_fit_W_gt_20} we get the values of table~\ref{table:Im_gn_vals_2}. These values are compatible with those in tables~\ref{table:GammaGamma_best_joint_fit_scaled_W_gt_10}, ~\ref{table:GammaGamma_best_joint_fit_scaled_W_gt_20} and~\ref{table:GammaProton_best_joint_fit_scaled_W_gt_10} but not with the ones of tables~\ref{table:SigmaProtonProton_best_fit_10_lt_W_lt_10000} and \ref{table:SigmaProtonProton_best_fit_10_lt_W_lt_386}.
\begin{table}[b!]
\centering
\caption{Values of the $\rm{Im} , g_n$ for the $\gamma^* \gamma$, $\gamma^* p$ and pp processes using the gravitational coupling values of table~\ref{table:GammaGamma_GammaProton_best_joint_fit_W_gt_10}.}
\vspace{0.5cm}
\begin{tabular}{|c|c|c|c|c|c|}
\hline
couplings   & value & couplins & value & couplings & value \\
\hline
${\rm Im} \, g_0^{\gamma^*\gamma}$  & -0.000168736 & ${\rm Im} \, g_0^{\gamma^*p}$ & -0.0498678 & ${\rm Im} \, g_0^{pp}$ & 241.064 \\ 
\hline
${\rm Im} \, g_1^{\gamma^*\gamma}$  & 0.000257243 & ${\rm Im} \, g_1^{\gamma^*p}$ & 0.0198756 & ${\rm Im} \, g_1^{pp}$ & 82.4686 \\ 
\hline
${\rm Im} \, g_2^{\gamma^*\gamma}$  & -0.0000331526 & ${\rm Im} \, g_2^{\gamma^*p}$ & -0.00468239 & ${\rm Im} \, g_2^{pp}$ & 77.9993 \\ 
\hline
${\rm Im} \, g_3^{\gamma^*\gamma}$  & 0.00153715 & ${\rm Im} \, g_3^{\gamma^*p}$ & 0.349152 & ${\rm Im} \, g_3^{pp}$ & 3276.88 \\ 
\hline
\end{tabular}
\label{table:Im_gn_vals_1}
\end{table}
\begin{table}[b!]
\centering
\caption{Values of the $\rm{Im} , g_n$ for the $\gamma^* \gamma$, $\gamma^* p$ and pp processes using the gravitational coupling values of table~\ref{table:GammaGamma_GammaProton_best_joint_fit_W_gt_20}.}
\vspace{0.5cm}
\begin{tabular}{|c|c|c|c|c|c|}
\hline
couplings   & value & couplins & value & couplings & value \\
\hline
${\rm Im} \, g_0^{\gamma^*\gamma}$  & -0.000167816 & ${\rm Im} \, g_0^{\gamma^*p}$ & -0.04958 & ${\rm Im} \, g_0^{pp}$ & 239.596 \\ 
\hline
${\rm Im} \, g_1^{\gamma^*\gamma}$  & 0.000260388 & ${\rm Im} \, g_1^{\gamma^*p}$ & 0.0204731 & ${\rm Im} \, g_1^{pp}$ & 86.4452 \\ 
\hline
${\rm Im} \, g_2^{\gamma^*\gamma}$  & -0.0000337101 & ${\rm Im} \, g_2^{\gamma^*p}$ & -0.00023297 & ${\rm Im} \, g_2^{pp}$ & 0.189894 \\ 
\hline
${\rm Im} \, g_3^{\gamma^*\gamma}$  & 0.00161256 & ${\rm Im} \, g_3^{\gamma^*p}$ & 0.352599 & ${\rm Im} \, g_3^{pp}$ & 3185.64 \\ 
\hline
\end{tabular}
\label{table:Im_gn_vals_2}
\end{table}


\section{Acknowledgments}


This research received funding from the Simons Foundation grants 488637  (Simons collaboration on the Non-perturbative bootstrap)
and from the  grant CERN/FIS-PAR/0019/2017. 
Centro de F\'\i sica do Porto is partially funded by Funda\c c\~ao para a Ci\^encia e a Tecnologia (FCT) under the grant
UID-04650-FCUP.
 AA is funded by FCT under the IDPASC doctorate programme with the fellowship  PD/BD/114158/2016.

\bibliographystyle{elsarticle-num}
\bibliography{photonScattering_notes}

\end{document}