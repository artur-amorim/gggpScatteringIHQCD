\documentclass[preprint, 12pt]{elsarticle}
\usepackage{amssymb}
\usepackage{amsfonts}
\usepackage{amsmath}
\usepackage{color}
\usepackage{ulem}
\usepackage[portuguese, english]{babel}
\usepackage[utf8]{inputenc}
\usepackage[T1]{fontenc}

\newcommand{\overbar}[1]{\mkern 1.5mu\overline{\mkern-1.5mu#1\mkern-1.5mu}\mkern 1.5mu}


\journal{Physics Letters B}

\begin{document}

\begin{frontmatter}

\title{Photon Scattering in Improved Holographic QCD}

\author[ath]{Artur Amorim~\corref{cor1}}
\author[ath]{Miguel S. Costa~\corref{cor1}}
\address[ath]{Centro de F\'{\i}sica do Porto e Departamento de F\'{\i}sica e Astronomia da Faculdade de Ci\^encias da Universidade do Porto, Rua do Campo Alegre 687, 4169-007 Porto, Portugal}

\cortext[cor1]{Corresponding author}

\begin{abstract}

This document is a set of notes that I have elaborated while using the holographic models developed by us to fit $\sigma\left(\gamma \gamma \rightarrow X\right)$, $F_2^\gamma\left(x, Q^2\right)$ and $\sigma\left(\gamma p \rightarrow X\right)$ data. The kernel parameters of the holographic models are fixed by the structure functions $F_2$ and $F_L$ of the proton while the fitting parameters are the gravitational couplings that appear in the vertices of the Witten diagram.

\end{abstract}

\end{frontmatter}

\section{$\gamma^{*}\gamma$ observables}
\begin{figure}[!h]
  \center
  \includegraphics[height=4cm]{images/Witten_diagram} 
  \caption{Tree level Witten diagram representing spin $J$    exchange in a $12\to34$ scattering. 
%     Double lines are used to denote graviton's propagator. The $n_1$ and $n_3$ labels denote incoming/outgoing proton's polarizations respectively, for the forward scattering $n_1=n_3$.
}
  \label{fig:Witten_diagram}
\end{figure}
In this section we will present a holographic computation of  $\sigma\left(\gamma \gamma \rightarrow X\right)$ and $F_2^\gamma\left(x, Q^2\right)$ in the context of the Improved Holographic QCD model. Although in these processes we have at most one off-shell photon involved we will first compute the general scattering amplitude $\gamma^{*}\left(Q_1\right)\gamma^{*}\left(Q_2\right) \rightarrow \gamma^{*}\left(Q_3\right) \gamma^{*}\left(Q_4\right)$. Later, from this general amplitude, we specify for the case of $\sigma\left(\gamma \gamma \rightarrow X\right)$ and $F_2^\gamma\left(x, Q^2\right)$. The Witten that is relevant for our calculations is the one in figure~\ref{fig:Witten_diagram}.


\subsection{Kinematics}
We start with kinematics first. Consider light-cone coordinates~$\left(+, -, \perp \right)$ with metric given by $ds^2 = - dx^{+} dx^{-} + d x_\perp^2$, where $x_\perp$ is a vector of the impact parameter space $\mathbb{R}^2$. The incoming photons have space-like momenta
\begin{equation}
k_1 = \left( \sqrt{s}, - \frac{Q_1^2}{\sqrt{s}}, 0 \right), \quad k_2 = \left( - \frac{Q_2^2}{\sqrt{s}}, \sqrt{s},  0 \right),
\end{equation}
while the outgoing off-shell photons have also the space-like momenta
\begin{equation}
k_3 = - \left( \sqrt{s},  \frac{q_\perp^2 - Q_3^2}{\sqrt{s}}, q_\perp \right), \quad k_4 = - \left(  \frac{q_\perp^2-Q_4^2}{\sqrt{s}}, \sqrt{s},  -q_\perp \right).
\end{equation}
The off-shellness of these 4-vectors is $k_i^2 = Q_i^2 > 0 \, \left( i = 1, \dots, 4\right)$. We consider the Regge limit of large s and fixed $t = -q_\perp^2$.

We can now define the photons polarization vectors $n^\lambda_i$ that satisfy the condition
\begin{equation}
n_i^\lambda \cdot k_i = 0.
\end{equation}
The incoming photons have the following polarizations
\begin{equation}
    n_1=
    \begin{cases}
      \left(0,0,1,0\right), & \lambda=1 \\
      \left(0,0,0,1\right), & \lambda=2 \\
      \frac{1}{Q_1} \left( \sqrt{s}, \frac{Q_1^2}{\sqrt{s}}, 0, 0 \right), & \lambda = 3
    \end{cases}
\end{equation}
\begin{equation}
    n_2=
    \begin{cases}
      \left(0,0,1,0\right), & \lambda=1 \\
      \left(0,0,0,1\right), & \lambda=2 \\
      \frac{1}{Q_2} \left( \frac{Q_2^2}{\sqrt{s}}, \sqrt{s}, 0, 0 \right), & \lambda = 3
    \end{cases}
\end{equation}  
and the outgoing off-shell photons have the following pollarization vectors
\begin{equation}
    n_3=
    \begin{cases}
      \left(0,\frac{2 q_x}{\sqrt{s}},1,0\right), & \lambda=1 \\
      \left(0,\frac{2 q_y}{\sqrt{s}},0,1\right), & \lambda=2 \\
      \frac{1}{Q_3} \left(\sqrt{s}, \frac{Q_3^2+q_\perp^2}{\sqrt{s}}, q_\perp \right), & \lambda = 3
    \end{cases}
\end{equation}  
\begin{equation}
    n_4=
    \begin{cases}
      \left(-\frac{2 q_x}{\sqrt{s}},0,1,0\right), & \lambda=1 \\
      \left(-\frac{2 q_y}{\sqrt{s}},0,0,1\right), & \lambda=2 \\
      \frac{1}{Q_4} \left(\frac{Q_4^2+q_\perp^2}{\sqrt{s}},\sqrt{s},- q_\perp \right), & \lambda = 3
    \end{cases}
\end{equation}
Notice that the transverse photons~$\left(\lambda=1,2\right)$ are normalized such that~$n^2 = 1$ while the longitudinal photons~$\left(\lambda=3\right)$ are normalized such that~$n^2 = -1$

\subsection{External states and Spin J fields}

The off-shell photons the external states are sources of the conserved U(1) current $j^\mu = \psi \gamma^\mu \psi$, where quark field $\psi$ is associated with the open string sector. According to the gauge/gravity duality dictionary, these current operators are dual to the non-normalizable modes of a massless~$U(1)$ gauge field A. We assume that this field comes from the open string sector and that it has a minimal coupling to gravity. Then it satisfies the following action 
\begin{equation}
S_A = - \frac{1}{4} \int d^5 X \sqrt{-g} e^{- \Phi} F_{ab} F^{ab}
\end{equation}
where $F=dA$ and we use the notation $X^a=(z,x^\alpha)$ for five-dimensional points. 
We will fix the gauge of the $U(1)$ bulk field to be $D_a A^a = 0$, which gives  
$A_z = 0$ and $\partial_\mu A^\mu = 0$. In this gauge, the solution to the equation of motion 
$ \nabla_a \left( e^{-\Phi} F^{ab} \right)=0$ is given by
\begin{equation}
 A_\mu^\lambda \left( X ; k\right) =  n_\mu^\lambda \, f_k ( z )\,e^{i k \cdot x}\,,
\end{equation}
where $f_k(z)$ solves the differential equation
\begin{equation}
  \label{eq:u1 eom gauge fixed}
  \left[-Q^2+e^{\Phi-A}\partial_z\left(e^{A-\Phi}\partial_z \right) \right]f_Q(z)= 0 \,.
\end{equation}
The momentum  $k$ and the polarisation vector $n^\lambda$ satisfy
\begin{equation}
  k^2 = Q^2 \,, \qquad 
  n^\lambda_z = 0 \,, \qquad
   k \cdot n^\lambda = 0 \,,
\end{equation}
where the boundary polarisation are given by the expressions of the last section. 
We choose as UV boundary condition $f(0)=1$ which gives the non-normalizable solution.
Later it will be useful to use the identities
\begin{align}
&F_{\mu\nu}\left(X; k, n\right) = 2 i k_{[\mu}n_{\nu]} f_Q \left(z\right) e^{i k \cdot x}, \notag \\
&F_{z\nu} \left(X; k, n\right) = n_\nu \dot{f}_Q e^{i k \cdot x}.
\label{eq:F_expression}
\end{align}

To compute the Witten diagram we also need to know the external scattering states couple with the spin J fields $h_{a_1 \dots a_J}$ in the graviton Regge trajectory. 
Furthermore, in the Regge limit we will only be interested in the TT components $h_{\alpha_1 \dots \alpha_J}$ of these fields. The derivation of this coupling has been done in~\cite{ballon_bayona_unity_2017} and hence we only present the result
\begin{equation}
  \kappa_J  \int  d^5 X \sqrt{-g} \, e^{-\Phi }   F^{\alpha_1 a} \partial^{\alpha_2} \dots \partial^{\alpha_{J-1}}F^{\alpha_J}_{\ \ a}
  h_{\alpha_1\dots \alpha_J} .
\end{equation}

\subsection{Computation of the Witten diagram}
The amplitude associated with the exchange of a spin J field in the Witten diagram is given by
\begin{align}
&A_J \left(k_i, \lambda_j\right) = - k_J^2 \int d^5X d^5 \bar{X} \sqrt{-g} \sqrt{-\bar{g}} e^{-\Phi} e^{-\bar{\Phi}} \times  \notag \\ 
&F_{- a}^{(1)} \left(k_1, z\right) \partial_{-}^{J-2}F^{a \, (3)}_{-} \left(k_3, z \right)  F_{+ b}^{(2)} \left(k_2, \bar{z}\right) \bar{\partial}_{+}^{J-2}F^{b \, (4)}_{+} \left(k_4, \bar{z} \right) \times \notag \\
& \Pi^{- \dots - , + \dots +}\left(X, \bar{X}\right)
\end{align}
Using the kinematics introduced and lowering the indeces of the bulk-to-bulk propagator of the spin J field we get
\begin{align}
&- k_J^2 {\left( \frac{i}{2} \sqrt{s}\right)}^{2J-4} \int d^5X d^5\bar{X} \frac{e^{5 (A+\bar{A}) }}{4} e^{-\Phi - \bar{\Phi}} {\left(4 e^{-2(A+\bar{A})}\right)}^J e^{-i q_\perp \cdot \left(x_\perp - \bar{x}_\perp\right)} \times \notag \\
& \times F_{-a}^{(1)}F^{a \, (3)}_{-} \Pi_{+\dots+, - \dots -}\left(X, \bar{X}\right)  F_{+ b}^{(2)}F^{b \, (4)}_{+}
\end{align}
By performing the change of variables $w = x - \bar{x}$, using the propagator identities
\begin{align}
&\int d^2l_\perp e^{-i q_\perp \cdot l_\perp} \int \frac{dw^+ dw^-}{2} \Pi_{+\dots+, - \dots -}\left(w^+, w^-, l_\perp, z, \bar{z}\right) = - \frac{i}{2^J} {\left( e ^{A+\bar{A}}\right)}^{J-1} G_J \left(z, \bar{z}, t\right)  \notag \\
& \int \frac{dx^+ dx^-}{2} d^2 x_\perp = {\left(2\pi\right)}^4 \delta^{(4)}\left(0\right) = V
\end{align}
and equation (\ref{eq:F_expression}) we get
\begin{align}
&\frac{i V k_J^2}{2^J} s^J \int dz d\bar{z} e^{3\left(A+\bar{A}\right)} e^{- \Phi - \bar{\Phi}} e^{-(1+J)(A+\bar{A})} F(1,3)F(2,4) G_J \left(z, \bar{z}, t\right), \notag \\
& F(i,j) = 
\begin{cases}
f_{Q_i} f_{Q_j} , \lambda_i = \lambda_j = 1,2 \\
\frac{\dot{f_{Q_1}}}{Q_i} \frac{\dot{f_{Q_j}}}{Q_j}, \lambda_i = \lambda_j = 3
\end{cases}, \notag \\
&G_J \left(z, \bar{z}, t\right) = e^{B+\bar{B}} \sum_n \frac{\psi_n\left(J, z\right) \psi_n \left(J, \bar{z}\right)}{t_n\left(J\right) - t},
\end{align}
where $B = \Phi - A/2$ in our models.

We now need to sum the above amplitude over the even spin-J fields with $J>2$. Such sum can be computed through a Sommerfeld-Watson transform
\begin{equation}
\frac{1}{2} \sum_J s^J + (-s)^J = -\frac{\pi}{2} \int \frac{dJ}{2 \pi i} \frac{s^J + (-s)^J}{\sin \pi J}.
\end{equation}
This assumes that an analytic continuation of the amplitude to the complex J-plane is possible. We now deform the integral from the poles at even J, to the poles $J = j_n\left(t\right)$ defined by $t_n(J) = t$. The scattering domain of negative t contains these poles along the real axis for $J<2$. The forward scattering amplitude ( $t= 0$ ) of $\gamma^{*}\left(Q_1\right)\gamma^{*}\left(Q_2\right) \rightarrow \gamma^{*}\left(Q_3\right) \gamma^{*}\left(Q_4\right)$ is then
\begin{align}
&\mathcal{A}^{\lambda_1, \lambda_2, \lambda_3, \lambda_4}_{Q_1, Q_2, Q_3, Q_4}\left(s,t = 0\right) = - \frac{\pi}{2} \sum_n s^{j_n\left(t\right)} \left( i + \cot\left(\frac{\pi j_n\left(0\right)}{2}\right) \right) \frac{k_{j_n\left(0\right)}^2}{2^{j_n\left(0\right)}} \frac{d j_n}{dt} \times \notag \\
& \times \int dz e^{-(j_n - 1.5) A} F(1,3) \psi_n\left(z\right) \int d\bar{z} e^{-(j_n-1.5)\bar{A}} F(2,4) \psi_n\left(\bar{z}\right)
\label{eq:scattering_amplitude}
\end{align}

From this expression of the scattering amplitude we can compute two observables: $F_2^\gamma$ and $\sigma\left(\gamma \gamma \rightarrow X\right)$.
\subsection{$F_2^\gamma$}
In high-energy $e^{+}e^{-}$ interactions some of the electrons and positrons do not annihilate or scatter electromagnetically, but are scattered by radiating virtual photons. These virtual photons can fluctuate in quark-antiquark pairs and radiate themselves gluons producing a hadronic final state X.


\begin{equation}
F_2^\gamma = \sum_n \frac{ Im \, g_n}{4 \pi^2 \alpha} Q^{2 j_n} x^{1- j_n} \int dz e^{- \left( j_n - 3/2\right) A} \left( f_Q^2 + \frac{\dot{f}_Q}{Q^2} \right) \psi_n \, ,
\end{equation}
where $g_n$ is defined by
\begin{equation}
g_n = - \frac{\pi}{2} \left( i + \cot \frac{\pi j_n}{2} \right) \frac{k^2_{j_n}}{2^{j_n}} \frac{d j_n}{dt} \int d\bar{z} e^{-\left(j_n - 3/2 \right)A} \psi_n
\end{equation}
\subsection{$\sigma\left(\gamma \gamma \rightarrow X\right)$}
In this process both photons are considered quasi-real
\begin{equation}
\sigma\left(\gamma \gamma \rightarrow X\right) = \sum_n Im \, g_n s^{j_n - 1} \int dz e^{- \left( j_n - 3/2\right) A}  \psi_n
\end{equation}
Here $g_n$ as the same definiton as in the $F_2^\gamma$ section.

\section{$\gamma\, p$ observables}

The total cross-section of the process $\gamma p \rightarrow X$ is related to the DIS structure function $F_2\left(x, Q^2\right)$ through
\begin{equation}
\sigma\left(\gamma p \rightarrow X\right) = 4 \pi^2 \alpha \lim_{Q^2 \rightarrow 0} \frac{F_2\left(x, Q^2\right)}{Q^2}.
\end{equation}
In IHQCD we have derived the following expressions for the structure functions
\begin{align}
&F_2(x, Q^2) = \sum_{n} \frac{ Im g_n}{4 \pi^2 \alpha} Q^{2 j_n} x^{1-j_n} \int dz \,e^{-\left(j_n-\frac{3}{2}\right)A}  \left( f_Q^2  +  \frac{\dot{f}_Q^{2}}{Q^2}      \right) \psi_n , \\
&F_L(x, Q^2) = \sum_{n} \frac{Im g_n}{4 \pi^2 \alpha} Q^{2 j_n} x^{1-j_n} \int dz \,e^{-\left(j_n-\frac{3}{2}\right)A}  \frac{\dot{f}_Q^{2}}{Q^2}  \psi_n.
\end{align}
Using the fact that $x = Q^2 / s$ and that for $Q = 0, \, f_Q\left(z\right) = 1$ our holographic expression for this cross-section is
\begin{equation}
\sigma\left(\gamma p \rightarrow X\right) =  \sum_{n} Im g_n \, s^{j_n -1 } \int dz \,e^{-\left(j_n-\frac{3}{2}\right)A}  \psi_n.
\end{equation}

\begin{equation}
g_n = - \frac{\pi}{2} \left( i + \cot \frac{\pi j_n}{2} \right) \frac{k_{j_n} \bar{k}_{j_n}}{2^{j_n}} \frac{d j_n}{dt} \int d\bar{z} e^{-\left(j_n - 7/2 \right)A} \upsilon_m^2 \psi_n
\end{equation}

\section{$\sigma\left(p p \rightarrow X\right)$}

\begin{equation}
\sigma\left(p p \rightarrow X\right) = \sum_n Im g_n \, s^{j_n -1}
\end{equation}

\begin{equation}
g_n = - \frac{\pi}{2} \left( i + \cot \frac{\pi j_n}{2} \right) \frac{\bar{k}^2_{j_n}}{2^{j_n}} \frac{d j_n}{dt} {\left(\int d\bar{z} e^{-\left(j_n - 7/2 \right)A} \upsilon_m^2 \psi_n\right)}^2
\end{equation}

\section{Data analysis}



\subsection{$\gamma^{*} \gamma$ processes}

\begin{table}[b!]
\centering
\caption{Values of the parameters for the fit with $\sigma\left(\gamma \gamma \rightarrow X\right)$ data with PHOJET unfolding. The total number of data points is 18 and a $\chi^2$ of 0.804515 was obtained.}
\vspace{0.5cm}
\begin{tabular}{|c|c|}
\hline
couplings   & value \\
\hline
$g_0$  & 0.00102487\\ 
\hline
$g_1$  & 0.00465463 \\ 
\hline
$g_2$  & 0.0340127  \\
\hline
$g_3$  & 0.0024513\\ 
\hline
\end{tabular}
\label{table:SigmaGammaGamma_best_fit_PHOJET}
\end{table}

\begin{figure}[!h]
  \center
  \includegraphics[width = \textwidth]{images/SigmaGammaGamma_vs_W_PHOJET_data_only.pdf} 
  \caption{Predicted $\sigma\left(\gamma \gamma \rightarrow X\right)$ vs experimental points using PHOJET unforlding. Our curve was obtained using the values from table ~\ref{table:SigmaGammaGamma_best_fit_PHOJET}.}
}
  \label{fig:SigmaGammaGamma_best_fit_PHOJET_data_only}
\end{figure}

\begin{table}[b!]
\centering
\caption{Values of the parameters for the fit with $\sigma\left(\gamma \gamma \rightarrow X\right)$ data with PHYTIA unfolding. The total number of data points is 18 and a $\chi^2$ of 0.927157 was obtained.}
\vspace{0.5cm}
\begin{tabular}{|c|c|}
\hline
couplings   & value \\
\hline
$g_0$  & 0.00194039 \\ 
\hline
$g_1$  & 0.00925169 \\ 
\hline
$g_2$  &  0.0738589 \\
\hline
$g_3$  &  0.00817258\\ 
\hline
\end{tabular}
\label{table:SigmaGammaGamma_best_fit_PHYTIA}
\end{table}

\begin{figure}[!h]
  \center
  \includegraphics[width = \textwidth]{images/SigmaGammaGamma_vs_W_PHYTIA_data_only.pdf} 
  \caption{Predicted $\sigma\left(\gamma \gamma \rightarrow X\right)$ vs experimental points using PHYTIA unfolding. Our curve was obtained using the values from table ~\ref{table:SigmaGammaGamma_best_fit_PHYTIA}.}
}
  \label{fig:SigmaGammaGamma_best_fit_PHYTIA_data_only}
\end{figure}

\begin{table}[b!]
\centering
\caption{Values of the parameters for the fit with $\sigma\left(\gamma \gamma \rightarrow X\right)$ data with the average of PHOJET and PYTHIA unfolding. The total number of data points is 18 and a $\chi^2$ of 0.711115 was obtained.}
\vspace{0.5cm}
\begin{tabular}{|c|c|}
\hline
couplings   & value \\
\hline
$g_0$  & 0.00173369 \\ 
\hline
$g_1$  & 0.008549 \\ 
\hline
$g_2$  &  0.0719086 \\
\hline
$g_3$  &  0.00825794\\ 
\hline
\end{tabular}
\label{table:SigmaGammaGamma_best_fit_Processed}
\end{table}

\begin{figure}[!h]
  \center
  \includegraphics[width = \textwidth]{images/SigmaGammaGamma_vs_W_Processed_data_only.pdf} 
  \caption{Predicted $\sigma\left(\gamma \gamma \rightarrow X\right)$ vs experimental points using the average result of PHOJET and PHYTIA unfolding. Our curve was obtained using the values from table ~\ref{table:SigmaGammaGamma_best_fit_Processed}.}
}
  \label{fig:SigmaGammaGamma_best_fit_Processed_data_only}
\end{figure}


\begin{table}[b!]
\centering
\caption{Values of the parameters for the fit with $F_2^{\gamma}\left(x, Q^2\right)$ data. The total number of data points is 22 and a $\chi^2$ of 1.00065 was obtained.}
\vspace{0.5cm}
\begin{tabular}{|c|c|}
\hline
couplings   & value \\
\hline
$g_0$  & 8.51615e-05\\ 
\hline
$g_1$  & 0.00071289 \\ 
\hline
$g_2$  & 0.000483367  \\
\hline
$g_3$  & -0.00316919\\ 
\hline
\end{tabular}
\label{table:F2Photon_best_fit}
\end{table}

\begin{figure}[!h]
  \center
  \includegraphics[width = \textwidth]{images/F2Photon_F2Photon_data_only.pdf} 
  \caption{Predicted $F_2^\gamma\left(x, Q^2\right)$ vs experimental points. Our curve was obtained using the values from table ~\ref{table:F2Photon_best_fit}.}
}
  \label{fig:F2Photon_best_fit_F2_data_only}
\end{figure}



\begin{table}[b!]
\centering
\caption{Values of the parameters for the joint fit of the proton structure functions $F_2^{\gamma}\left(x, Q^2\right)$ and $\sigma\left(\gamma \gamma \rightarrow X\right)$ data with PHOJET unfolding for the later. The total number of data points is 40 and a $\chi^2$ of 1.37416 was obtained.}
\vspace{0.5cm}
\begin{tabular}{|c|c|}
\hline
couplings   & value \\
\hline
$g_0$  & 0.000157591\\ 
\hline
$g_1$  & 0.000299143 \\ 
\hline
$g_2$  & 0.000137773  \\
\hline
$g_3$  & -0.00181583\\ 
\hline
\end{tabular}
\label{table:GammaGamma_best_fit_PHOJET}
\end{table}

\begin{figure}[!h]
  \center
  \includegraphics[width = \textwidth]{images/F2Photon_PHOJET.pdf} 
  \caption{Predicted $F_2^\gamma\left(x, Q^2\right)$ vs experimental points. Our curve was obtained using the values from table ~\ref{table:GammaGamma_best_fit_PHOJET}.}
}
  \label{fig:F2Photon_best_fit_PHOJET}
\end{figure}

\begin{table}[b!]
\centering
\caption{Values of the parameters for the joint fit of the proton structure functions $F_2^{\gamma}\left(x, Q^2\right)$ and $\sigma\left(\gamma \gamma \rightarrow X\right)$ data with PHYTIA unfolding for the later. The total number of data points is 40 and a $\chi^2$ of 2.32196 was obtained.}
\vspace{0.5cm}
\begin{tabular}{|c|c|}
\hline
couplings   & value \\
\hline
$g_0$  & 0.000171144\\ 
\hline
$g_1$  & 0.00021032 \\ 
\hline
$g_2$  & 0.00015387  \\
\hline
$g_3$  & -0.0015843\\ 
\hline
\end{tabular}
\label{table:GammaGamma_best_fit_PHYTIA}
\end{table}

\begin{figure}[!h]
  \center
  \includegraphics[width = \textwidth]{images/F2Photon_PHYTIA.pdf} 
  \caption{Predicted $F_2^\gamma\left(x, Q^2\right)$ vs experimental points. Our curve was obtained using the values from table ~\ref{table:GammaGamma_best_fit_PHYTIA}.}
}
  \label{fig:F2Photon_best_fit_PHYTIA}
\end{figure}

\begin{table}[b!]
\centering
\caption{Values of the parameters for the joint fit of the proton structure functions $F_2^{\gamma}\left(x, Q^2\right)$ and $\sigma\left(\gamma \gamma \rightarrow X\right)$ data with the average result of PHOJET and PHYTIA unfolding for the later. The total number of data points is 40 and a $\chi^2$ of 1.70487 was obtained.}
\vspace{0.5cm}
\begin{tabular}{|c|c|}
\hline
couplings   & value \\
\hline
$g_0$  & 0.000166002\\ 
\hline
$g_1$  & 0.000247537 \\ 
\hline
$g_2$  & 0.000112687  \\
\hline
$g_3$  & -0.00169226\\ 
\hline
\end{tabular}
\label{table:GammaGamma_best_fit_Processed}
\end{table}

\begin{figure}[!h]
  \center
  \includegraphics[width = \textwidth]{images/F2Photon_Processed.pdf} 
  \caption{Predicted $F_2^\gamma\left(x, Q^2\right)$ vs experimental points. Our curve was obtained using the values from table ~\ref{table:GammaGamma_best_fit_Processed}.}
}
  \label{fig:F2Photon_best_fit_Processed}
\end{figure}



\subsection{$\gamma \, p$ processes}

Here we present the fits results.

\begin{table}[b!]
\centering
\caption{Values of the parameters for the joint fit of the proton structure functions $F_2\left(x, Q^2\right)$ and $F_L\left(x, Q^2\right)$ and $\sigma\left(\gamma p \rightarrow X\right)$ data with $x \leq 0.01$, $Q^2 \leq 400 \, \text{GeV}^2$ and $\sqrt{s} > 1.89 \, \text{GeV}$. This means a $\chi^2 / \text{N.d.f.}$ of 1.76745 with 392 experimental points.}
\vspace{0.5cm}
\begin{tabular}{|c|c|}
\hline
couplings   & value \\
\hline
$g_0$  & 0.0488725\\ 
\hline
$g_1$  & 0.0269148 \\ 
\hline
$g_2$  & 0.00283532  \\
\hline
$g_3$  & -0.377856\\ 
\hline
\end{tabular}
\label{table:GammaP_best_fit}
\end{table}

\begin{table}[b!]
\centering
\caption{Values of the parameters for the joint fit of the proton structure functions $F_2\left(x, Q^2\right)$ and $F_L\left(x, Q^2\right)$ and $\sigma\left(\gamma p \rightarrow X\right)$ data with $x \leq 0.01$, $Q^2 \leq 400 \, \text{GeV}^2$ and $\sqrt{s} > 5.93 \, \text{GeV}$. This means a $\chi^2 / \text{N.d.f.}$ of 1.41597 with 345 experimental points.}
\vspace{0.5cm}
\begin{tabular}{|c|c|}
\hline
couplings   & value \\
\hline
$g_0$  & 0.0496542\\ 
\hline
$g_1$  & 0.0215857 \\ 
\hline
$g_2$  & -0.00273519  \\
\hline
$g_3$  & -0.356926\\ 
\hline
\end{tabular}
\label{table:GammaP_best_fit_2}
\end{table}

Here are the plots

\begin{figure}[!h]
  \center
  \includegraphics[scale = 0.75]{images/SigmaGammaP_vs_W.pdf} 
  \caption{Our curve vs experimental points. Our curve was obtained using the values from table ~\ref{table:GammaP_best_fit}.}
%     Double lines are used to denote graviton's propagator. The $n_1$ and $n_3$ labels denote incoming/outgoing proton's polarizations respectively, for the forward scattering $n_1=n_3$.
}
  \label{fig:GammaP_best_fit}
\end{figure}

\begin{figure}[!h]
  \center
  \includegraphics[width = \textwidth]{images/Holographic_F2_splitted.pdf} 
  \caption{$F_2\left(x,Q^2\right)$ vs experimental points. Our curve was obtained using the values from table ~\ref{table:GammaP_best_fit}.}
%     Double lines are used to denote graviton's propagator. The $n_1$ and $n_3$ labels denote incoming/outgoing proton's polarizations respectively, for the forward scattering $n_1=n_3$.
}
  \label{fig:F2_best_fit}
\end{figure}

\begin{figure}[!h]
  \center
  \includegraphics[width = \textwidth]{images/Holographic_FL_splitted.pdf} 
  \caption{$F_L\left(x,Q^2\right)$ vs experimental points. Our curve was obtained using the values from table ~\ref{table:GammaP_best_fit}.}
%     Double lines are used to denote graviton's propagator. The $n_1$ and $n_3$ labels denote incoming/outgoing proton's polarizations respectively, for the forward scattering $n_1=n_3$.
}
  \label{fig:FL_best_fit}
\end{figure}


\begin{figure}[!h]
  \center
  \includegraphics[scale = 0.75]{images/SigmaGammaP_vs_W_2.pdf} 
  \caption{Our curve vs experimental points. Our curve was obtained using the values from table ~\ref{table:GammaP_best_fit_2}.}
%     Double lines are used to denote graviton's propagator. The $n_1$ and $n_3$ labels denote incoming/outgoing proton's polarizations respectively, for the forward scattering $n_1=n_3$.
}
  \label{fig:GammaP_best_fit_2}
\end{figure}


\begin{figure}[!h]
  \center
  \includegraphics[width = \textwidth]{images/Holographic_F2_splitted_2.pdf} 
  \caption{$F_2\left(x,Q^2\right)$ vs experimental points. Our curve was obtained using the values from table ~\ref{table:GammaP_best_fit_2}.}
%     Double lines are used to denote graviton's propagator. The $n_1$ and $n_3$ labels denote incoming/outgoing proton's polarizations respectively, for the forward scattering $n_1=n_3$.
}
  \label{fig:F2_best_fit_2}
\end{figure}

\begin{figure}[!h]
  \center
  \includegraphics[width = \textwidth]{images/Holographic_FL_splitted_2.pdf} 
  \caption{$F_L\left(x,Q^2\right)$ vs experimental points. Our curve was obtained using the values from table ~\ref{table:GammaP_best_fit_2}.}
%     Double lines are used to denote graviton's propagator. The $n_1$ and $n_3$ labels denote incoming/outgoing proton's polarizations respectively, for the forward scattering $n_1=n_3$.
}
  \label{fig:FL_best_fit_2}
\end{figure}




\bibliographystyle{elsarticle-num}
\bibliography{bib/pomeron.bib, bib/refs.bib, bib/AdSCFT.bib, bib/glueballs.bib, bib/HQCD.bib bib/exp_data.bib}
\end{document}
